\chapter[Conclusion]{Conclusion}
\label{chap:conclusion}
\epigraph{\fixspacing\emph{Take it from me, there's nothing like a job
    well done. Except the quiet enveloping darkness at the bottom of a
    bottle of Jim Beam after a job done any way at all.}}{Hunter
    S. Thompson}

%Maybe it meant something. Maybe not, in the long run, but no
%explanation, no mix of words or music or memories can touch that
%sense of knowing that you were there and alive in that corner of time
%and the world

% Leave space between title and quote or publication note.  This has often been
% 10cm for a quote and 8 cm for a reference, but this is really up to you.
%\vspace{8cm}

%\vfil\eject\clearpage
\clearpage
NGC 891 offers a unique opportunity to perform stellar population
analysis and compare the results to our detailed understanding of the
distribution of stars in the Milky Way. In this way it occupies an
important bridge between the Milky Way and surveys that offer a large
sample size, but a smaller spatial resolution. The closeness and
nearly totally edge-on nature of NGC 891 also allows for unambiguous
determination of finely sampled vertical gradients in stellar
population, which makes it a perfect test-case for theories concerning
the origin of disk stratification seen in the Milky Way.

To take advantage of the information available in NGC 891 I helped
design and construct HexPak/\GP, a pair of fiber IFUs for the WIYN
telescope. In Chapter \ref{chap:pak_build} I detail the design and
fabrication of these instruments. The important features of HexPak/\GP
are:
\begin{enumerate}
\item HexPak has a standard hexagon shape made mostly of fibers with
  an on-sky diameter of 2\farcs8. It also has a high resolution core
  of 18 0\farcs94 fibers that make it ideally suited for studies of
  face-on galaxies or any bright object where high spatial resolution
  is desired.

\item \GP is roughly rectangular in shape and has five different fiber
  sizes ranging from 1\farcs87 - 5\farcs62. The regions of different
  size are arranged in a gradient that is optimized to measure roughly
  exponentially decreasing surface brightness at roughly
  \val{10}{Mpc}; a similar S/N per fiber can be achieved in a single
  exposure. \GP is ideally suited for observations of objects with a
  large dynamic range in brightness where spectral resolution can be
  sacrificed for observing efficiency.

\item At the spectrograph input the slits of HexPak and \GP share a
  foot and focal plane. This allows observers to swap between the two
  IFUs with zero modifications to the Bench Spectrograph.

\item The head fixtures of HexPak and \GP, while physically separate,
  share a common focal plane in the WIYN IAS. This further eases the
  transition between the two IFUs; in practice an observer can switch
  between HexPak and \GP during an observing run in roughly 10
  minutes.

\end{enumerate}

During the conception and construction of HexPak/\GP I researched ways
to improve the optical performance of fiber-based instruments, mainly
through the mitigation of FRD. In Chapter \ref{chap:FRD} I detail my
experiments, which are broadly applicable to all fiber optic systems,
and find:
\begin{enumerate}

\item FRD is dominated by light entering the fiber at smaller angles
  (i.e., closer to the axis of light propagation).

\item A secondary component of FRD is attributable to the end-polish
  of fiber surfaces. FRD decreases with polishing down to finer grit
  sizes, but not significantly below grit-sizes of 5\mum.

\item Total throughput also depends on end-polish, with a wavelength
  dependence that indicates the increase in throughput is simply a
  reduction in surface-scattering.  The most significant gains occur
  for polishing that proceeds down to 5 $\mu$m grit, although for most
  astronomical applications at low light-levels polishing finer than
  this level is measurably advantageous.

\item The amount of FRD does \textbf{not} depend on wavelength.

\end{enumerate}

Chapter \ref{chap:891_1} details the first set of results from a
program that measures stellar populations in NGC 891 with \GP. In this
chapter I detail challenges in data acquisition/reduction caused by
the unique nature of \GP, but also present methods that largely
eliminate any negative impact on the resulting data.

I also describe the observing program that is designed to cover NGC
891 out to large heights and radii. The design of \GP is ideally
suited to a program of this nature and allows for efficient collection
of high-quality data.

I then use the well known and characterized LICK spectral index system
to identify separate stellar populations in NGC 891 and find:
\begin{enumerate}

\item There is a clear transition with height above NGC 891's disk
  midplane between young and old populations at \val{0.4}{kpc}
  (roughly the broad-band exponential scale-height). This is
  consistent with models of heating of the stellar disk in the solar
  cylinder.

  \item For $|z| > \val{0.4}{kpc}$ there is a trend towards younger
    populations at larger projected radii, consistent with an
    inside-out formation history in NGC 891. The trend also suggests a
    flaring of the young stellar disk at radii beyond 8 kpc.

  \item Beyond 8 kpc in radius and between 0.4 kpc $\leq |z| <$ 1 kpc
    there is a a clear asymmetry in age between the two sides of the
    galaxy. The approaching side, where there is more H$\alpha$
    emission, appears younger. This can be explained by spiral
    structure in NGC 891; on the approaching side of the galaxy we see
    the leading edge of a spiral arm that has very recent/ongoing star
    formation. Our sight-lines to the receding side of the galaxy,
    however, look onto the trailing edge of another arm that is
    obscured by high concentrations of dust.

\end{enumerate}

Finally, in Chapter \ref{chap:891_2} I employ the power of
full-spectrum fitting to get a detailed and quantitative view of
stellar populations in NGC 891. To confidently interpret the results
of this method I need to understand the interplay between all of the
fit parameters, namely the well known degeneracies between age,
metallicity, and extinction. Furthermore, assumptions about the star
formation history in NGC 891 can introduce large systematics in our
results.

The degeneracies between age, metallicity, and extinction are
exacerbated by SSP template libraries that constitute a large set of
individual SSPs with little thought to the astrophysical similarities
that exist over a wide range of age and metallicity values. In other
words, most SSP template libraries have many SSPs that, while assigned
different age or metallicity values, have very similar spectra. To
mitigate these degeneracies I employ diffusion k-means to create a new
SSP basis set that greatly reduces the number of SSP templates while
still preserving important astrophysical features. With this new SSP
basis set I estimate the uncertainty in fit parameters caused by
degeneracies between the model spectra to be roughly 10\% for age and
extinction and \asim 20\% for metallicity.

I also quantify the systematic age uncertainty that arises from
assuming a star formation history during the interpretation of the
fitting results. In the worst case these uncertainties are \asim 20\%,
but I argue that the worst case (totally random star formation on
small physical scales) is unrealistically pessimistic for our
data. NGC 891 is a coherent galaxy, and over long timescales (\asim 1
Gyr) the star formation history should be relatively the same across
the entire galaxy. Thus the systematic uncertainties do not apply when
comparing detailed structure within NGC 891; they are only important
when comparing NGC 891 to other galaxies that may have different star
formation histories.

With an understanding of uncertainties in my results I identify
details in the structure and formation history of NGC 891. Namely,
\begin{enumerate}

\item The picture of inside-out galaxy formation seen in Chapter
  \ref{chap:891_1} is enhanced by the view that each generation of
  stars forms a flared disk, with later generations forming disks with
  larger radii.

\item Two distinct epochs in the enrichment history of NGC 891. During
  the first few Gyr of its life NGC 891 existed in a somewhat
  closed-box state; from an initial seed of pristine gas each
  generation of stars formed from gas enriched by the generations
  before it. However, about \val{6}{Gyr} ago new, metal-poor gas
  from outside the galaxy started to fall onto the disk of NGC
  891. This new gas lowered the metallicity of stars formed in the last
  \val{6}{Gyr} and the indication is that metallicity will continue
  to decrease in the future.

\item Three groupings of stellar populations in the
  ($r,|z|,\tau_L,Z_L,A_V$) parameter space: a primary disk, a flared
  extension of the disk, and a metal-rich sequence at large radii and
  heights. These groupings are a consequence of the formation and
  enrichment history presented in the previous point. The primary disk
  is the main disk of the galaxy and is composed of multiple mono-age,
  flared disks that increase in radius with galactic age. The
  ``flare'' at large radii is corresponds to the most recent batch of
  star formation. The third sequence is the portion of the disk that
  formed during peak metallicity \val{\asim 6}{Gyr} ago.

\end{enumerate}

This work constitutes one of the first detailed, resolved studies of
stellar populations in a nearby galaxy, and as such lays the groundwork
for future studies. In particular, the methods outlined here could
easily be applied to other nearby, edge-on galaxies. This thesis
provides a robust set of tools, from instruments to data reduction and
analysis methods, that will allow future astronomers to rapidly expand
our view of stellar populations and disk formation.

\acknowledgements{The work presented in this thesis was directly
  supported by the U.S. National Science Foundation (NSF) grants 
  ATI-0804576, AST-1009471, and AST-1517006.}


\clearpage
\phantomsection % Fixes references link in hyperref/PDF index

\chapter[Stellar Populations in NGC 891]{The Location of Stellar Populations in NGC 891}
\label{chap:891_pop}

% Leave space between title and quote or publication note.  This has often been
% 10cm for a quote and 8 cm for a reference, but this is really up to you.
%\vspace{8cm}

%%%%%%%%%%%%%%%%%%%%%%%%%%%%%%%%%%%%%%%%%%%%%%%%%%%%%%%%%%%%% 
%% \begin{chabstract}
%%     Chapter abstract.
%% \end{chabstract}
%% \cleardoublepage

\section{Outline}
Kind of the ``meat'' of the thesis. Here we go in depth about the analysis
methods used on the data described in \ref{chap:gradpak_obs}. The punchline is
some statement about how age (and, to a lesser precision, metallicty) varies
with radius and height in NGC 891.

\section{Basic Analysis}
\begin{itemize}
  \item Velocities
  \item Emission Correction
  \item Extinction Model
\end{itemize}
\subsection{Schedule}
This is basically done. The two things that still need work are 1) deciding
where exactly the extinction model section goes, and 2) basic editing. This
could be completed in 2 days.

\section{LOS Depths}
\begin{itemize}
  \item Velocity-based measurement
  \item Optical depth based measurement
\end{itemize}
\subsection{Schedule}
Great progress has been made over the last few days on this front. I think the
velocity stuff is done and 80\% written up.

The optical-depth section of this has a completed framework, but the analysis
depends on a lot of assumptions about the dust distribution that we will take
from other sources. This requires more thought. Still, a week should be
sufficient to complete this section.

\section{SSP Fitting}
\begin{itemize}
  \item Basic model
  \item Chisq weights
  \item Model Libraries
  \item Determining Age/Metallicity
\end{itemize}
\subsection{Schedule}
The first three bullet points above are basically done and need only basic
copy editing. This could be done in a few days.

The last bullet point is where we have been spending most of our time recently
and I think we are extremely close to being ``done'' with the actual
analysis. I mean, shit, we could say ``we're going to use weights because they
exclude known bad metallicities, and we're going to use this power because
we've sampled a coarse grid of powers and it does the best in terms of
rejecting metallicities, but not too many metallicities'' tomorrow and be done
with this. The pipline to produce ages and metallicities already exists, we're
just iterating on the uncertainties.

Once the anlysis is done this will take a week to write up.

\section{Spectral Indices}
\begin{itemize}
  \item Measurement
  \item Use as independent check on metallicities used in SSP fitting
  \item Qualitative results of trends in age, metallicity with height
\end{itemize}
\subsection{Schedule}
Some of this is already done. The last bullet point needs the most work. In
particular I have not settled on exactly what plane of spectral indices are
the most informative. Trager and Co. like Balmer vs <MgFe> and Balmer vs <Fe>,
but the latter of those does not work well with our galaxy models (dynamic
range is very small). We currently use \Hd vs <MgFe> and \Hd vs Mgb, but the
justification is lacking.

I think if we stay relatively shallow on this topic we can get everything
figured out relatively quickly. Maybe 1-2 weeks.

\section{Heating Models}
\begin{itemize}
  \item Model recipe
  \item comparison to data
\end{itemize}
\subsection{Schedule}
The ball is mostly in your court on this one. We have ages as a function of r
and z, now we need models for comparison. In principle this could be a very
short, ``look, we're not crazy'' section with an eye towards future, in depth
work.

\bibliographystyle{thesis}
\bibliography{891_pop}

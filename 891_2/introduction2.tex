\section{Introduction}

In Chapter \ref{chap:891_1} we used a focused set of spectral line
measurements to show that the signature of disk heating in NGC 891
matches observations of the Milky Way \citep[e.g.,][]{Bovy12,
  Hayden15} with startling accuracy. Near the midplane of NGC 891
($|z| < \val{0.4}{kpc}$) spectra have strong Balmer absorption
features, indicative of young stellar populations, while above
\val{0.4}{kpc} there is a sharp transition to spectra dominated by
metal abosportion, notably Ca H\&K, MgI, and Fe. Secondary results
include detection of a negative age gradient with radius, which could
be the signature of an inside-out formation history similar to that
seen in the Milky Way \citep{Bovy12, Hayden15}, and a potential flare
in the star-forming disk beyond \val{\asim 8}{kpc}.

{\bf MAB: In above paragraph I would say features associated Ca H\&K,
  the G band, Mgb and Fe bands 5270 and 5335.}

These results represent important new information about NGC 891 (and,
by extension, Milky Way-like disk galaxies), but they do not take
advantage of the full wealth of information contained in our
data. Specifically, the index measurements presented in
\ref{chap:891_1} consider each spectral feature in isolation
(or, at best, in relation to one other feature) and ignore the fact
that every spectral channel encodes information about the underlying
astrophysical pictures. By construction, the data taken by \GP and
described in \ref{chap:891_1} are of sufficiently high
signal-to-noise (S/N) that using the entire spectrum can yield
physical insights beyond what can be measured by line indices alone.

In this chapter we extend our previous results by using full-spectral
fitting to measure age, metallicity, and extinction (our three
``quantities of interest'') as a function of radius and height in NGC
891. As discussed in Chapter \ref{chap:intro} full spectrum fitting
requires an SSP basis set, which in turn depends on stellar template
library, IMF, and stellar isochrones. In \S\ref{891_2:sec:SSP_sets} we
explore a range of different SSP libraries and methods for mixing them
together to form a full model galaxy.

The primary reason for this exploration of different SSPs and methods
is a desire to understand and quantify uncertainties that arise from
astrophysical degeneracies in the SSP libraries. It is well known
\citep{Oconnel76,Aaronson78,Worthey94,dePaz02} that stars of different
metallicities and ages can have spectral shapes that are very similar,
and this similarity poses a challenge to interpreting the results of a
full spectrum fit to a mix of SSPs; it is very difficult to know which
of the many similar SSPs represents the ``correct'' astrophysical
reality. Confounding this issue further is the fact that the presence
of dust also affects the final galaxy spectrum. The result is a
complex covariance between age, metallicity, and extinction (i.e.,
dust).

Some modern fitting codes, like FIREFLY \citep{Wilkinson15} or STEKMAP
\citep{Ocvirk06}, attempt to circumvent the impact of dust by first
removing low-order information from both the model SSPs and galaxy
data. The advantage of this method is that, by effectively removing
the continuum shape, the goodness-of-fit is driven primarily by the
strength of stellar absorption lines\footnote{ In
  \S\ref{891_2:sec:smart_chi} we make our own attempt to increase the
  power contained in strong spectral lines, with limited
  success.}. Ironically this is a step towards focused spectral index
measurements like those employed in Chapter \ref{chap:891_1}, albiet
one with a much larger set of lines and all the advances of modern
computing behind it. However, this increased focus on absorption lines
requires sacrificing information contained in the continuum shape and,
perhaps more importantly, ignores degeneracies that exist between,
e.g., extinction and age.

In this chapter we use a somewhat simple method to perform the galaxy
fits (see \S\ref{891_2:sec:SSP_method}). The strength of our
astrophysical interpretation comes from our relatively high S/N data
(see Chapter \ref{chap:891_1}) and our understanding and
quantification of the uncertainties in our fits
(\S\ref{891_2:sec:uncertainty}).

%% We choose to use the SSP templates of \citet{Bruzual03},
%% which are constructed with the Padova isochrones \citep{Bertelli94,
%%   Girardi00, Marigo08}, Chabrier IMF \citet{Chabrier03}, and STELIB
%% library \citep{LeBorgne03}.

%% The idea of using an
%% entire galaxy spectrum has a long history since \citet{Spinrad71}
%% first mixed stars together and modern techniques employ a variety of
%% sophisticated methods to extract the maximum about of data from each
%% wavelength. Regardless of specific method, all attempts at
%% full-spectrum fitting require the same basic ingredients: (i) a
%% library of stellar spectra, whether empirical or synthetic, (ii) a set
%% of isochrones that encapsulate how stars evolve with time, and (iii)
%% an initial mass function (IMF). With these three components
%% Astronomers can construct simple stellar populations (SSPs); a set of
%% stars of the same age and same metallicity/abundance with a mass
%% distribution determined by the IMF. To simulate an entire galaxy
%% multiple SSPs of different ages and metallicities are combine together
%% to produce a complext stellar population (CSP), which requires
%% assuming both a star formation history (SFH) and the
%% distribution/properties of dust in the galaxy. An excellent discussion
%% of this process can be found in \citet[especially his Figure
%%   1]{Conroy13}.

%% Within the general picture painted above there exists a wide range of
%% options and data. It is common for SSP libraries to be constructed
%% with the Padova isochrones \citep{Bertelli94, Girardi00, Marigo08}
%% because these models cover the widest range of stellar age and
%% chemical compositions, but other models are often used for their focus
%% on specific epochs of stellar evolution, for example high-mass stars
%% (Geneva \citep{Schaller92,Meynet00}), low-mass stars ($Y^2$
%% \citep{Yi01,Yi03}, or Dartmouth \citep{Dotter08}), and even very
%% low-mass stars (Lyon \citep{Chabrier97,Baraffe98}). In this work we
%% use exclusively the Padova isochrones because our observations, by
%% their very nature, light-weighted and therefore the specific details
%% of low mass stars are rather unimportant.

%% The choice of IMF can also affect the final modeled galaxy
%% spectrum. The canonical IMF of \citet{Salpeter55} was based on
%% observations of the Solar Cylinder in the Milky Way, and there is so
%% far little evidence that the IMF is appreciably different elsewhere in
%% the Universe \citep{Bastian10}. More recent observations have refined
%% the specific form of the IMF \citep{Kroupa01, Chabrier03}, but the
%% general picture remains the same. We use the IMF of \citet{Chabrier03}
%% because it is physically motivated and provides a good fit to low-mass
%% and brown dwarf star counts in the Milky Way
%% \citep{Bruzual03,Chabrier01,Chabrier03}.

%% Finally, the construction of SSPs depends on the stellar library
%% used. In this work we consider only empirical stellar libraries, and
%% more specifically on the STELIB \citep{LeBorgne03} and MILES
%% \citep{Sanchez-Blazquez06} libraries. The main strength of emperical
%% libraries is that the get the chemistry right by default, as indeed
%% they must. The cost of this accuracy, however, is a very limit
%% sampling of the entire parameter space of stellar evolution. For
%% example, as is discussed in \S\ref{891_2:sec:ma11}, the MILES library very
%% coarsly samples the metallicty/age plane and doesn't have any spectra
%% for ages below \val{6.5}{Myr}. We use the STELIB library because it
%% more finely samples stars of different ages, but warn that it still
%% lacks a detailed view of how spectra change with metallicity.

%% As discussed in \S\ref{891_2:sec:SSP_sets} we ultimately use the SSPs of
%% \citet{Bruzual03}, which are constructed with the Padova isochrones,
%% Chabrier IMF, and STELIB library. We note, however, that over the
%% wavelength range we consider ($\val{3800}{\AA} \leq\lambda\leq
%% \val{6800}{\AA}$) the differences in spectral shape caused by
%% different assumptions/models are minimal.

%% More directly relevant to our work is the assumption about the SFH
%% that is used to construct galaxies (CSPs) from SSPs. A common choice
%% for SFH is the so called $\tau$-model where the star formation rate
%% (SFR) follows an exponential function with a single scale parameter,
%% $\tau_\mathrm{SF}$. This analytic form is based on closed-box models
%% where the SFR depends linearly on gas density \citep{Schmidt59} and
%% offers an attractive, one parameter, parameterization of the SFH. In
%% this work we chose to use a non-parametric SFH (as discussed in
%% \S\ref{891_2:sec:SSP_method}) which allows us to reduce the systematics in
%% our results that arise from forcing the form of the SFR (systematics
%% are not completely eliminated, however, as discussed in
%% \S\ref{891_2:sec:sys_err}). Using this method results in a much larger set
%% of free parameters and puts more strain on the fitting code, but our
%% data have high enough signal to noise (\val{\asim 60}{\AA^{-1}}) to
%% make it a viable option.

%% Once model galaxies are constructed there are a multidue of methods
%% available to fit them to our data, for example those of
%% \citep{Cappellari04,Tojeiro07,Chen12,CidFernandes05,Ocvirk06,Wilkinson15,Sanchez16}. Regardless
%% of the method used these fits face the same common issues; namely how
%% to deal with known degeneracies between age, metallicity, and
%% extinction \citep{Oconnel76,Aaronson78,Worthey94,dePaz02}. In some
%% methods the extinction degeneracy can be mitigated by removing the
%% overall continuum from both the data and models before fitting
%% \citep[e.g.,][]{Ocvirk06,Wilkinson15} and then recovering an
%% extinction estimate either by measurements of gass emission (i.e., the
%% Balmer decrement) or separate analysis of the ``best fit'' galaxy
%% spectrum. Metallicity and age are more closely entwined and the
%% degeneracy between them more difficult to break. In this work we
%% attempt to quantify the uncertainties that arise from similarities
%% between SSPs that are degenerate with age and metallicity (see
%% \S\ref{891_2:sec:fit_err}), but note that we have not addressed systematic
%% uncertainties that may arise from our choice of model SSPs.

%% % {\bf Discussion of history
%% %   of SSP fitting}

%% % o PPXF (Cappelari 2004)
%% % o VESPA (Tojeiro 2007)
%% % o PCA (Chen 2012)
%% % o STARLIGHT (Cid Fernandes 2005)
%% % o STECKMAP (Ocvirk 2006)
%% % o FIREFLY (Wilkinson 2015?)

%% % {\bf MAB: History of SPS fitting is a big job. You might take a look
%% %   at Conroy's relatively recent ARAA article, and think about what you
%% %   want to focus on. One key issue is the move away from indices; this
%% %   is what folks have been doing for a while now. It might be worth
%% %   noting why we used indices in Paper I. The specific issues
%% %   associated with SPS and full spectral fitting (FSF) revolve around

%% % (1) completeness of stellar libraries;

%% % (2) stellar evolutionary tracks (isochrones), particularly for late
%% % phases of stellar evolution; and

%% % (3) fitting methods.

%% % Your task here should be to mention (1) and (2), and note that while
%% % you will be taking a look at two different set of libraries and tracks
%% % (BC03 and Maraston), in the wavelengths of interest there are not
%% % likely to be major differences. The more significant issue for you is
%% % in the fitting methods. Here I would further break this issue down 
%% % in terms of addressing the following question:

%% % How do codes handle the degeneracies in age and metallicity, and, more
%% % broadly, age, metallicity and extinction?

%% % We know about Firefly in terms of splitting off extinction. However,
%% % what is less clear is how various codes *pre*-select the set of
%% % templates to minimize degeneracy, and how they explore possible
%% % parameter space.  You are going to need to take some time to look at
%% % the literature on this. Pipe3D (Sebastian Sanchez); Starlight (Cid
%% % Fernandez); Vespa (Rita Tojeiro); and whatever Patricia Blazquez does.}

In addition to its importance as a nearby Milky Way analog, NGC 891
offers a unique perspective as one of the few galaxies close enough
for detailed study that is also perfectly edge-on. This orientation
allows for unambiguous measurements of vertical trends, but also
provides an opportuntiy to measure a fully three dimensional picture
of stellar populations. Unlike face-on disk galaxies, edge-on disk
galaxies have a clear and strong velocity structure along the line of
sight that is caused by the rotation of the galaxy. The detailed
kinematic structure of NGC 891 is known
\citep{Swaters97,Kregel05,Oosterloo07} and we can leverage this
knowledge to determine the true location of measured data within the
galaxy. In principle a detailed study of the kinematics (i.e., line
shape) of an edge-on galaxy like NGC 891 could yield a very detailed
picture of the that galaxy's line of sight (LOS) structure (what we
call ``doppler tomography''), but in this work we are restricted by
spectral resolution to simply measuring line centroids.

%+++++++++++++++++++++++++++++++++ {\bf MAB: I'd say more here about
% your preference for edge-ons for getting the 3D perspective. For
% example, the emission density profile (from stars or gas) along the
% line of sight, convolved with the velocity field, produces a line
% profile shape. This is what we mean by Doppler tomography; it
% requires some assumptions about the velocity field, which we take to
% be smooth and similar to the atomic gas, and about the smoothness of
% the density distribution of the emisssion, which was problematic for
% the ionized gas but probably a better bet for the older stars
% contributing to the metal-line absorption.  All that said, i.e., put
% out the idea, we use the line centroid only because of our limited
% resolution.}

The following chapter is organized as follows: in
\S\ref{891_2:sec:data} we briefly describe the data of
\ref{chap:891_1}, which is used in this work. \S\ref{891_2:sec:anal}
details further analysis needed to ready the data for SSP fitting,
along with our general fitting methodology. In \S\ref{891_2:sec:LOS}
we use velocity information to determine the true, 3D location of our
data in NGC 891, and in \S\ref{891_2:sec:SSP_sets} we discuss a range
of different SSP libraries and motivate our final choice of library
used for fitting. An attempt to improve the precision of our fits by
modifying $\chi^2$ is shown in \S\ref{891_2:sec:smart_chi} while in
\S\ref{891_2:sec:uncertainty} we quantify both fitting and systematic
uncertainties present in our analysis. Finally,
\S\ref{891_2:sec:results} presents a new view of the phase-space of
NGC 891 as revealed through our full spectrum fits and we summarize
this work in \S\ref{891_2:sec:summary}.

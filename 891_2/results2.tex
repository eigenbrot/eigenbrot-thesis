\section{Results}
\label{891_2:sec:results}

\subsection{Morphological Features in NGC 891}
\label{891_2:sec:rz}

The analysis methods described above yield a parameter space
consisting of $\tau_L$, $Z_L$, and $A_V$ as a function of
\emph{galactic} radius and height. Examining NGC 891 in this parameter
space reveals three distinct features: (i) a ``primary'' disk, (ii) a
flared extension of the same disk, and (iii) a high-metallicity
sequence that exists at large radii, which we call simply ``sequence
3''. Figures \ref{891_2:fig:MLWA_rz} - \ref{891_2:fig:MLWZ_rz} show these
features and we discuss how we identified them based on three
projections of this parameter space below.

%++++++++++++++++=
%% {\bf MAB: what do you mean by a ``phase space'' ? Since we might
%% want to consider phase space as the canonical 6D V-X manifold  from
%% statistical mechanics, perhaps just ``parameter space''?

%% I would not use the description or label in the figures of '``normal''
%% star forming disk.' You can call it sequence-1, or 'primary disk,' but
%% it is not all star-forming. In fact, I would conclude from the radial
%% and vertical distribution that this 'primary disk'' includes the old
%% inner disk and the current star-forming disk.

%% Somewhere up front you need to discuss you have identified these three
%% popoulations, or that you will describe how you arrived at it in some
%% later section.

%---------------
%% A basic question: Why  are we looking at (age,Z,A) vs r and z instead
%% of plotting r vs z and color-coding by age or Z or A?

%% ADE comment: Whaa? I don't get this comment. By plotting both XX vs
%% r and XX vs z side by side we mitigate the amount of unique
%% information encoded in color bars, which are generally more
%% difficult to parse than location on a plot. Making the only
%% encoding of the values of interest (age, metallicty, extinction) be
%% in color would make these plots very hard to read.

%% }

\subsubsection{{\Large $\tau_L$}}
\begin{figure*}
  \centering
  \includegraphics[width=\textwidth]{891_2/figs/MLWA_rz_all.pdf}
  \caption[$\tau_L$ vs
    ($r,|z|$)]{\fixspacing\label{891_2:fig:MLWA_rz}$\tau_L$ as a
    function of true radius (\emph{left}) and height
    (\emph{right}). On each side the points are color coded by the
    opposite position parameter. Open and filled circles correspond to
    the approaching ($r_\mathrm{proj} < 0$) and receding
    ($r_\mathrm{proj}\geq 0$) side of NGC 891, respectively. Vertical
    dotted lines correspond to the radial and vertical cuts used in
    \ref{chap:891_1} and Figure \ref{891_2:fig:SFH_cuts}. The three
    rows show the exactly the same data, but shaded to highlight the
    feature identified in the right-hand labels.}
\end{figure*}

Figure \ref{891_2:fig:MLWA_rz} shows the projection of our data onto the
$\tau_L(r,z)$ plane. The three broadly defined regions from
\ref{chap:891_1} are immediately obvious in the top-right
panel (i.e., primary disk): below \val{0.4}{kpc} all of the
populations are young (\val{< 6}{Gyr}), the region where $0.4 < |z|
\leq \val{1}{kpc}$ is a transition zone where the age generally
increases, and above \val{1}{kpc} the average population age
asymptotes to the oldest age (\val{\asim 10}{Gyr}). As described in
\S\ref{891_2:sec:sys_err} the specific value of $\tau_L$ at the largest
heights is dependent on the physical assumptions made when
constructing the models (i.e., the first stars in the Universe formed
\val{\asim 12}{Gyr} ago), but these assumptions only modulate the
actual value; the conclusion that the integrated light is dominated by
the oldest possible populations above \val{1}{kpc} is valid regardless
of the specific population age.

\citet{Xilouris99} fit a single stellar disk to broad band photometric
profiles of NGC 891 and find a scale height of 0.38 - \val{0.43}{kpc}
from $K$ to $B$ bands. In this context it is plausible that our
primary disk roughly corresponds to this single stellar disk. However,
more recent studies of NGC 891 \citep{Schechtman-Rook12,
  Schechtman-Rook13, Schechtman-Rook14} have refined this single-disk
mode to include three distinct disk components: super-thin, thin, and
thick with scale heights (in $K_S$ band) of 0.16, 0.47, and
\val{1.44}{kpc}, respectively. In this view it is likely that the
smooth transition from young to old ages seen in the upper-left panel
of Figure \ref{891_2:fig:MLWA_rz} is a result of the superposition of these
different disk components and thus our primary ``disk'' encapsulates
substructure beyond our measurement capabilities. Furthermore, the
primary disk also does not extend much beyond $r=\val{8}{kpc}$, which
is where \citet{Schechtman-Rook13} identify a truncation in their
super-thin disk

%++++++++++
%% {\bf MAB: Xilouris99 is a fine place to start, but I would like you to
%%   move on from discussing the Xilourin99 single-disk fit to the more
%%   sophisticated modeling from ASR (cite all 3 papers). With that in
%%   mind you should consider the different radial zones and how the
%%   flare fits in with that.

%++++++++++
%% I don't agree with your statement that above 1 kpc we are seeing the
%% halo as opposed to the thick disk. The points I would like to see
%% emphasized are that (1) there is a smooth trend of increasing age with
%% height that plateaus above 1 kpc. (2) At intermediate heights there is
%% considerable variation in age that you will then proceed to explain by
%% the flare and sequence-3. (3) That the 'primary disk' or sequence-1
%% appears to be limited primarily to the inner 8 kpc, corresponding to
%% the radial break of the super-thin disk seen by ASR. The thicker
%% component seems to be radially shorter and quite old suggesting it is
%% associated with the old inner disk and pseudo bulge. The thinner
%% portion has a range of ages. As a whole, the upper age envelop of the
%% 'primary disk' or sequence-1 appear to drop rapidly beyond 6 kpc.

%% }

Radial trends in $\tau_L$ reveal detailed substructure in NGC 891. The
top-left panel of Figure \ref{891_2:fig:MLWA_rz} confirms that, in the
primary disk at a given radius, older populations occupy larger
heights, but it is also clear that the maximum age at these large
heights decreases slightly with increasing radius.

%--------------
%% {\bf MAB: I agree generally with the age-radius trend except sequence-3
%% kind of mucks this up. Thoughts?}

The middle row of Figure \ref{891_2:fig:MLWA_rz} highlights the flare of NGC
891. At low heights it occupies the same $\tau_L$ space as the main,
primary disk, but as height increases the ages stay relatively young
compared to the main disk. Importantly almost all of the apertures
belonging to this sequence exist at radii beyond \val{8}{kpc}, which
is one of the main reasons we identify this feature as a flared
extension of the star forming disk. Such a flare would be manifest as
an increase in scale height with radius and thus the star-forming disk
would extend to larger heights at large radii. Given this simplistic
picture, and assuming the star-forming disk extends to roughly 1 scale
height at all radii we can roughly estimate that at radii \val{>
  8}{kpc} flaring has increased the scale height by roughly a factor
of two (up to \val{\asim 0.8}{kpc}).

%% {\bf MAB: how quantitative is your scale-height estimate? This is
%%   nice, and if it is robust we ought to quote it in the abstract.}
%% ADE comment: not very robust, hence the change to ``roughly estimate''

The bottom row of Figure \ref{891_2:fig:MLWA_rz} shows the high-metallicity
``sequence 3''. It exists only at heights above \val{0.4}{kpc} and
mostly beyond $r = \val{8}{kpc}$ and is made up of stellar populations
that are generally younger than the stars from the primary disk
in the same ($r,z$) locations. That said it is clear that many of the
apertures in sequence 3 contain a significant amount of light
contributed by Old population stars with the slightly lower age at large
heights is most likely indicative of an increased contribution from I2
populations.

%++++++++++++=
%% {\bf MAB: ``O population stars'' is very confusing because I read this
%%   as ``O stars,'' i.e., hot young stars not old DFK. So we have to be
%%   careful.}

% beyond these general trends there is evidence that at larger radii ($r
% > \val{\asim 8}{kpc}$) there is a separate grouping of points that
% occupy a younger locus than the rest of the data at similar
% radii. This seconday population still exhibits a general ageing of
% stellar population with height, but at values systematically below the
% main trend. Given the clumping of this younger grouping in both $r$
% and $z$ it is reasonable to assume that it occupies a coherent
% sub-structure in NGC 891. A flared disk has been suspected {\bf REF!}
% and would explain the presence of young stellar populations at larger
% heights. 

\subsubsection{{\Large $A_V$}}

\begin{figure*}
  \centering
  \includegraphics[width=\textwidth]{891_2/figs/AV_rz_all.pdf}
  \caption[$A_V$ vs ($r,|z|$)]{\fixspacing\label{891_2:fig:AV_rz}$A_V$
    as a function of true radius (\emph{left}) and height
    (\emph{right}), plotted in the same style as Figure
    \ref{891_2:fig:MLWA_rz}}
\end{figure*}

In Figure \ref{891_2:fig:AV_rz} the three morphological features mentioned
above are identified in the $A_V(r,z)$ plane. The primary disk (top
row) shows slightly increasing extinction with height below \val{\asim
  0.4}{kpc} followed by a roughly exponential decrease with height
above \val{\asim0.4}{kpc}. These results lend further proof to the
conclusion that the region below \val{0.4}{kpc} is the location of
current star formation, which requires regions of optically thick dust
and gas. Above \val{0.4}{kpc} the older stars are likely to be
smoothly distributed in a way that is uncorrelated with the dust. Thus
at the heights where old stars dominate we measure a reasonable
estimate of some mean extinction along a significant line of sight. At
low heights the dust is patchy, very dense, and correlated with
younger stars and so our line of sight is likely to be abruptly
truncated at some relatively short line of sight, and hence the stars
we \emph{do} see have relatively little extinction, hence the lower
extinction values below \val{0.4}{kpc}.

%----------------
%% {\bf MAB: Do you mean to say that $A_V(r)$ is an exponential
%%   function of $r$ above 0.2 kpc? Draw or fit a curve? This is the
%%   place to compare $A_V$ with the Balmer values. It would be
%%   particularly interesting to see if the different groups (primary,
%%   flare, seq-3) are different in this comparison. More on the
%%   radial trend in the next par.}
%% ADE: Not sure what you mean by $A_V(r)$ being exponential above 0.2 kpc.

%+++++++++++++++
%% {bf MAB: Here, concerning the down-turn at very low heights, let me
%%   say this differently and perhaps more plausibly. The down-turn at
%%   very low radii seems implausible, but we need to consider the
%%   transition from young to old populations seen in the previous
%%   figure. The older stars are likely to be smoothly distributed in a
%%   way that is uncorrelated with the dust. When the old stars dominate
%%   at larger heights we get a reasonable estimate of some mean
%%   extinction along a significant line of sight. At low height, since
%%   the dust is patchy, very dense, and correlated with younger stars,
%%   our line of sight is likely to be abruptly truncated at some
%%   relatively short line of sight, and hence the stars we do see have
%%   relatively little extinction. This should correspond to the LOS
%%   depths, which is why it would be nice to show the trend with height
%%   to refer back to here.

%% }

Above \val{0.4}{kpc} the general decline in $A_V$ is consistent with
the simple morphological view of exponentially decreasing surface
density in edge-on galaxies. The rate of decline shown in Figure
\ref{891_2:fig:AV_rz} indicates the scale height of attenuating material is
\val{\asim 0.6}{kpc}, which is significantly larger than either
\citet{Xilouris99} or \citet{Schechtman-Rook12}, who find a vertical
dust scale height of 0.3 and \val{0.24}{kpc} in the $V$ and $K_S$
bands, respectively. It is important to note, however, that both of
these studies find a wavelength dependence on extinction that is
steeper than the model of \citet{Charlot00} used in our model galaxies
(see \S\ref{891_2:sec:extinction}). Our prescription of extinction therefore
requires larger normalization values ($A_V$) to achieve the same level
of extinction which makes direct comparison of our data to the
conclusions of \citet{Xilouris99} or \citet{Schechtman-Rook12}
difficult.

%++++++++++++++
%% {\bf MAB: This is nice. Finish the analysis, include the results from ASR,
%% use all of the dust scale-heights for the different bands and let's
%% see if we can work this out. Show a model curve.}

The middle row of Figure \ref{891_2:fig:AV_rz} shows that flared extension
of the primary disk has systematically lower extinction than the main
disk. In a flared disk with a roughly constant total dust fraction an
increase in scale height at large radii (i.e., the flare) would cause
the dust surface density to decrease at these larger radii, which is
consistent with the lower values of $A_V$ seen in the data
corresponding to the flare. A paucity of data points above $z =
\val{0.4}{kpc}$ and within $r = \val{8}{kpc}$ makes it difficult to
determine vertical or radial trends in the flare's extinction, but to
1st order these trends appear to be constant.

%--------------
%% {\bf MAB: My read of Figure \ref{891_2:fig:AV_rz} is that there is evidence
%%   for a decrease in extinction at lower heights at larger radii.}
%% ADE: I don't agree with this

Sequence 3 does not stand out very much in the $A_V(r,z)$ planes. It
shows extinction values that are generally consistent with the other
morphological features that exist at smaller radii, albeit with more
scatter (bottom right panel of Figure \ref{891_2:fig:AV_rz}).

%++++++++++++++++++
%% {\bf MAB: My read of Figure \ref{891_2:fig:AV_rz} is that it is hard to
%%   conclude on seq-3 except to see that it appears to have much more
%%   variation in $A_V$ with height, although it does not appear to be
%%   inconsistent at a given height with its counterpart populations at
%%   smaller radius. I think you say much of this and I don't think it is
%%   important to emphasize that seq-3 ``occupies a locus with lower
%%   extinction than either the star forming disk or the flare'' because
%%   I think this can be explained by height.}

\subsubsection{{\Large $Z_L$}}

\begin{figure*}
  \centering
  \includegraphics[width=\textwidth]{891_2/figs/MLWZ_rz_all.pdf}
  \caption[$Z_L$ vs
    ($r,|z|$)]{\fixspacing\label{891_2:fig:MLWZ_rz}$Z_L$ as a function
    of true radius (\emph{left}) and height (\emph{right}), plotted in
    the same style as Figure \ref{891_2:fig:MLWA_rz}}
\end{figure*}

Finally, Figure \ref{891_2:fig:MLWZ_rz} shows $Z_L$ as a function of
location in NGC 891 and highlights the discriminatory power of
metallicity in identifying the three morphological features mentioned
above. The primary disk shows slightly negative correlations
between metallicity and both radius and height (top panel). This view
is consistent with measurements from the Milky Way \citep{Bovy12,
  Hayden14} and with the conclusions of
\ref{chap:891_1}. This also reaffirms the idea that stars at
large heights are formed from pristine gas at early times (as shown in
Figure \ref{891_2:fig:MLWA_rz}.

In the middle row of Figure \ref{891_2:fig:MLWZ_rz} shows strong evidence
that the flare identified in previous sections comes from the same
underlying distribution as the primary disk. At all radii and heights
in the $Z_L(r,z)$ plane the flare points are essentially
indistinguishable from those of the primary disk. The flare exists in
a different ($r,z,A_V$) location, but it's ages and, importantly,
metallicity are consistent with the same populations seen in the disk.

%% {\bf MAB: don't push too hard on Z for young-age pops.}  

Sequence 3 is perhaps most visible as a high-metallicity sequence in
the $Z_L(r,z)$ planes, as shown in the bottom row of Figure
\ref{891_2:fig:MLWZ_rz}. At all radii and heights it occupies a locus that
has larger values of $Z_L$ than either the primary disk or the flare
(which are the same in terms of metallicity). Despite these high
values it still follows a general trend of decreasing metallicity with
height and radius, which suggest that whatever evolutionary mechanisms
cause the trends in the disk also affect the large radii and heights
occupied by sequence 3.

%% These results, coupled with those shown in Figure \ref{891_2:fig:MLWA_rz}
%% yield a possible explanation for the presence of sequence 3. This
%% sequence is distinct in both age and metallicity from the primary disk
%% and flare

%+++++++++++++++
%% {\bf MAB: Isn't $Z_L(r,z)$ a volume not a plane? Maybe another way to
%%   say your last point which also ties in to the flare is that we are
%%   seeing a population sequence at large radii and large heights that
%%   is distinct in both age and metallicity from what looks to be the
%%   old thick disk seen over a range of radii. It may be a temporal
%%   extension of the flare seen at younger ages, i.e., stars that formed
%%   in a flared disk at an intermediate era.}

%% ADE: This is nice, but how does the temporal fare extension picture
%% jive with \emph{higher} metallicities than the younger flare?

% The average metallicity appears to increase slightly with
% height up to \val{1}{kpc} where it abruptly begins to decrease with
% increasing height. We note, however, that the apparant increase up to
% \val{1}{kpc} is driven mostly by a few super-solar points between 0.4
% and \val{1}{kpc} and at large radii. As we have seen above, the radial
% structure of NGC 891 is highly non-uniform and we therefore caution
% that these few points are probably not representative of a global
% trend. Thus ignoring these super-solar points the average metallicity
% is relatively constant below \val{1}{kpc}, at which point it decreases
% with height. This general trend is qualitatively consistent with
% results from the Solar cylinder in the Milky Way
% \citep{Hayden15,Bovy12} and supports the theory that large heights are
% primarily populated by old stars that formed from pristine gas.

% Much like $\tau_L$ and $A_V$ there appears to be a second order,
% bi-modal $Z_L$ distribution in NGC 891. Starting at $z = \val{\asim
%   0.4}{kpc}$ there is a cluster of apertures with super-solar
% metallicities that follow the same general trend as the global average
% (i.e., decreasing with radius above \val{1}{kpc}), but at a higher
% metallicity. All of these high-$Z_L$ apertures exist at radii beyond
% \val{8}{kpc}, which suggests they belong to the same sub-structure
% identified in $\tau_L$. Under the assumption that this sub-structure
% is a flared disk the presence of high metallicity stellar populations
% indicate that this disk is likely the site of a recent epoch of star
% formation, a claim supported by the systematically lower ages see in
% Figure \ref{891_2:fig:MLWA_rz} for the same structure.

\subsection{Star Formation History}
\label{891_2:sec:SFH}
\begin{figure*}
  \centering
  \includegraphics[width=\textwidth]{891_2/figs/SFH_cuts.pdf}
  \caption[SSP light weights in ($r,|z|$)
    grid]{\fixspacing\label{891_2:fig:SFH_cuts}Fractional light-weight
    (Equation \ref{891_2:eq:light_weight}) for each DFK age bin as a
    function of radius and height. The radial and height bins shown
    are the same as used in \ref{chap:891_1} and highlight the
    different components of NGC 891 described in the text. The
    horizontal extent of each bar spans the range of ages covered by
    each SSP and the vertical width of each bar corresponds to the
    fitting uncertainty of each weight computed using the same method
    detailed in \S\ref{891_2:sec:fit_err}. Each weight is color coded
    by the average $Z_L$ in the particular ($r,z$) bin.  }

\end{figure*}

A primary benefit of full-spectrum SSP fitting is that it allows us to
fit and measure the star formation history (SFH) for every aperture in
our data set. As discussed in \S\ref{891_2:sec:sys_err} the mean,
light-weighted age ($\tau_L$) is a convenient proxy for SFH, but
requires an assumed SFH for the models used to fit our galaxy data. To
avoid any assumptions we examine the integral of the SFH multiplied by
the flux of each SSP, which is identical to the light-weights found
during SSP fitting (see \S\ref{891_2:sec:SSP_method} and Equation
\ref{891_2:eq:MLWA}). Figure \ref{891_2:fig:SFH_cuts} shows these values in the
same grid of radius and height used in \ref{chap:891_1}
(although now the radius values are the true, cylindrical values a la
\S\ref{891_2:sec:LOS}). From these data we can see the same general trend
found in \ref{chap:891_1} and above: younger populations
predominantly exist below \val{0.4}{kpc}. There are also small
contributions from young populations above \val{0.4}{kpc} at the
largest radii, which is once again indicative of a potential flare in
NGC 891.

Above \val{0.4}{kpc} (and even below this for $r < \val{8}{kpc}$) the
I2 populations contribute more to the total light at larger radii,
with a corresponding decrease in the contribution from Old
populations. This trend is broadly consistent with an inside-out view
of star formation in NGC 891. Such a formation scenario (which we also
see in \ref{chap:891_1}) a higher fraction of young and
intermediate aged stars at large radii, exactly as seen in Figure
\ref{891_2:fig:SFH_cuts}.

%% Further evidence for a flare can be seen in the fact that I2
%% populations (see Table \ref{891_2:tab:dfk}) contribute more to the total
%% light at larger radii, with a corresponding decrease in contribution
%% from O populations. This picture provides further insight into
%% secondary morphological feature (i.e., flare) identified in
%% \S\ref{891_2:sec:rz}, specifically that, while it shows the same positive
%% age/height gradient seen in the main primary disk, its oldest
%% stellar populations are at most \val{\asim5.7}{Gyr} old (i.e., the old
%% limit of the I2 DFK bin). 

%% {\bf MAB: I am not sure I followed this paragraph. It is not clear to
%% me that the trends in O and I2 DFKs in the 6 bins for $z>0.4$ kpc
%% are consistent with flares. What if the thick disk is simply
%% younger at larger radii. Maybe I am missing something?}

%% Despite this decrease O populations are still the primary source of
%% light for all radii at heights above \val{1}{kpc}. This observation is
%% consistent with the conclusions from \S\ref{891_2:sec:rz} that any flare in
%% NGC 891 extends up to \val{\asim 1}{kpc}, but not much beyond this
%% height.

%++++++++++++++++++++++
%% {\bf MAB: In this discussion I think we need to be careful what we
%%   mean by flare. For example, the Martig+14 simulations have all of
%%   the mono-age populations flaring, just at different radii, inside
%%   out. So when we say flare, do we mean only the Y and I1 pops?}

%% ADE: Agree, the increase in I2 with radii can more simply be
%% explained as inside out formation, which is consistant with paper1

At first glance the $Z_L$ values shown in Figure \ref{891_2:fig:SFH_cuts}
seem counterintuitive; at the lowest heights the largest values of
metallicity occur in the oldest SSPs and the I2 SSPs show metallicity
values that increase with height. The first trend is likely due to the
positive correlation between age and metallicity identified at low
heights in \S\ref{891_2:sec:fit_err}. Indeed, below \val{0.4}{kpc} the Y and
I1 SSPs are very metal poor, while I2 and O have higher metallicities,
consistent with the degeneracies seen at these low heights. In other
words, the high metallicity in Old populations below \val{0.4}{kpc} is
likely a result of fitting degeneracies and perhaps not an accurate
representation of the physical picture in these regions.

%++++++++++++
%% {\bf MAB: This hi-Z O population at low heights has me concerned.  Do
%%   we think this is real or an artifact of our fitting?  It could be
%%   real if there is indeed an old, metal-rich thin disk, although one
%%   wonders why I2 is so metal poor at the same height. Since we have
%%   supposedly corrected for projection and have everything in the right
%%   radial bins it would be hard to invoke the older stars being from
%%   smaller radii where metallicity is higher. So this is why I wanted
%%   to see mass-weighted plots.}

To explain the trend in increasing metallicity with height seen in the
I2 populations we invoke the flare identified in this and previous
sections. Recent star formation in the ``normal'' primary disk (i.e.,
not the flare) does not extend much above \val{\asim 1}{kpc} and any
stars at these large heights are likely to be old and metal poor. The
flare, on the other hand, as shown above, exhibits recent (in this
case \val{\asim 6}{Gyr} ago) star formation at heights above
\val{1}{kpc}. Thus, as height increases the fraction of I2 stars that
belong to the flare (and therefore have higher metallicities)
increases and the average $Z_L$ in the I2 bin increases. This
interpretation is consistent with the morphological picture present in
\S\ref{891_2:sec:rz} and allows us to actually infer something about the SFH
in NGC 891. As shown in Table \ref{891_2:tab:dfk} the I2 bin spans roughly
\val{5}{Gyr} which is a significant fraction of the total age of NGC
891 (assumed here to be \val{12}{Gyr}). In the primary disk I2 stars
are considered ``old'', have low metallicities, and were therefore
likely formed at the older end of the I2 bin (\val{\asim 6}{Gyr}). At
larger heights, however, light from the flare dominates in all but the
oldest SSP bin, and thus the I2 DFK bin represents a fundamentally
different population of stars; those that formed near the young limit
of the I2 bin (i.e., \val{\asim 500}{Myr}) and therefore have higher
metallicities (as seen in Figure \ref{891_2:fig:SFH_cuts}).

%----------------
%% {\bf MAB: I don't understand the first sentence. Do you mean the
%%   positive metallicity gradient with radius for I2 at each height? Or
%%   do you mean the positive gradient in WL with radius for all heights?
%%   Or ...?  I would drop the claim that stars above 1 kpc are
%%   ``primordial halo stars that are coeval with NGC 891.'' You just
%%   don't know that. If you want to assume it, and then see what follows
%%   that's a different matter. In any evet I would not assume they are
%%   part of the halo.

%% I'm not sure I buy this and what follows: ``Thus, as height increases
%% the fraction of I2 stars that belong to the flare (and therefore have
%% higher metallicities) increases and the average $Z_L$ in the I2 bin
%% increases.'' Again, why is this all due to the flare as opposed to an
%% outer thick disk what has different age and metallicity properties
%% because of some inside-out formation?

%% ADE: I see what you're saying about ``why flare and not just a thicker
%% disk'', but I really think that Figure \ref{891_2:fig:MLWA_rz} tips the
%% scales in favor of a flare. Basically, the fact that our ``flare''
%% only exists at large radii rules out a thicker disk. Right? 
%% }


%% Given the data shown in Figure \ref{891_2:fig:MLWA_rz} we conclude that the
%% majority of the flare stars are coveal with the main star forming
%% disk, which suggests the flare is simply an increase in scale height
%% (of both gas and stars) at larger radii and not representative of a
%% fundamentally different population. Figure \ref{891_2:fig:MLWZ_rz}, however,
%% suggests that the flare stars \emph{do} occupy a location in parameter
%% space separate from the primary disk; one of higher
%% metallicity. The true picture is likely somewhere in the middle;
%% what's certain is that stars in the ``flare'' have a similar SFH as
%% the main primary disk but formed from gas with a higher
%% metallicity than that found towards the inner regions of NGC 891.

%++++++++++++++++++
%% {\bf MAB: We should talk about this. I don't have my head wrapped
%%   around it so I am not yet onboard.}

%% ADE: Ya, I don't get it either. Not sure what I was thinking
%% here. Figure 21 clearly shows that the ``flare'' stars have the
%% same metallicity as the primary disk. Just took it out for now.

%% The first order effect that we are seeing is that there is a positive
%% age gradient with height above the mid-plane, consistent with
%% qualitative expectations.

%% Lower metallicity (sub-solar) models appear to give flatter age
%% profiles at large height, which is what we would expect.  In contrast,
%% models with solar and above metallicity yield lower ages at the
%% largest heights relative to mid-latitude values, i.e., the ages roll
%% over a bit at large heights.  This may very well be a systematic
%% effect if indeed the high-latitude population is intrinsically
%% low-metallicity since forcing higher metallicity increases
%% line-strength which would then have to be compensated for in the model
%% fitting with younger ages. On the other hand, we need to be careful
%% (a) not to over-interpret small changes in age at large ages (where
%% there is little leverage in the models; and (b) not to make
%% conclusions based on our initial assumptions.

%% We don't see much correlation of the optical depth with the adopted
%% metallicity of the SPS models. We should quantify this, but it is
%% reassuring. This means, for example, that metallicity and age are
%% largely keying off the line-strengths and not the continuum spectral
%% shape, but we should check this in detail--for example, by looking at
%% correlations between age vs $A_V$ in residuals about the mean at a
%% given height.

%% There do appear to be radial trends in the sense that the outer disk
%% is younger, and younger at a given height, although this trend has not
%% been isolated from metallicity effects (e.g., perhaps the outer disk
%% is also more metal poor, as seen in the Milky Way).

%% We also have not looked carefully to determine if the age gradients
%% are symmetric with radius on either side of the galaxy.

\section{Summary}
\label{891_2:sec:summary}

We have used full-spectrum SSP fitting to measure trends in $\tau_L$,
$Z_L$, and $A_V$ with both radius and height in NGC 891. To mitigate
the impact of degeneracies between these quantities we downselect the
full \citetalias{Bruzual03} SSP template library in two ways: limiting
the range of metallicity values used, and reducing the age resolution
of the template library. We limit the range of allowed metallicities
to be $Z \geq 0.2\Zsol$ and show in \S\ref{891_2:sec:multi_metal} that
SSPs with lower metallicity produce model galaxies that; a) are
statistically worse than higher metallicities, in a $\chi^2$ sense,
and b) result in the astrophysically implausible situation of very old
super-solar metallicity stars combined with very young super metal
poor stars.

The age resolution of our SSP template libraries is reduced using the
statistically motivated method of diffusion K-means
\citepalias{Mosby15}, which accounts for the fact that there is very
little difference in spectral shape between old SSPs over a range of
roughly \val{5}{Gyr}. We find that this limited basis set still
accurately reproduces our observed galaxy spectra
(\S\ref{891_2:sec:final_SSP}).

While these steps do reduce the magnitude of degeneracies between
$\tau_L$, $Z_L$, and $A_V$ these degeneracies are still the dominant
sources of uncertainty in our fit parameters. However they are
typically on the order of \asim 10\%, which still allows us to measure
trends in these quantities with radius and height. We also find two
distinct regimes of degeneracy between $\tau_L$ and $Z_L$; for
intermediate and old populations these two quantities are negatively
correlated (as expected from e.g.,
\citet{Oconnel76,Aaronson78,Worthey94,dePaz02}) but for young
populations $\tau_L$ and $Z_L$ are positively correlated, indicating
that older populations have higher metallicities. This correlation is
caused by the interplay between all three of our fit parameters, as
discussed in \S\ref{891_2:sec:fit_err}, underlining the importance of
parameter covariance in understanding fit uncertainties.

%++++++++++++++++++++
%% {\bf MAB: In the above two paragraphs I think you want to say a bit
%%   more about the motivation behind reducing the number of templates
%% in the context of degeneracies and systematic errors.

%% Specifically, you have reduced the {\it range} of metallicity, and you
%% have reduced the {\it resolution} in age. You should say that both are
%% important, that you should specify that you have done both, and give
%% what that reduction is (metallicities cut; reduction to 4 age bins
%% with characteristic ages of [t1,...t4] Gyr via DFK). Then you should
%% say that limiting the metallicity range is crucial because of fitting
%% degeneracies that lead to astrophysically implausible stellar mixes of
%% very young and ultra metal-poor stars mixed with very old and super
%% metal-rich stars. Reducing the age resolution is particularly
%% important at older ages where there is little to no leverage from the
%% full spectral fitting to discriminate between 5 Gyr. These reductions
%% allow you to minimize and quantify the degeneracies between age and
%% metallicity and quantify the systematic uncertainties in the mean age,
%% as you will describe in the next paragraph.

%% I would not refer to the two degeneracy regimes by the physical
%% location in the galaxy, but by the characteristic age of the stellar
%% populations (see the request to do this earlier in the relevant
%% section). This is more fundamental and allows other studies to use the
%% information directly. You then can mention that these characteristic
%% age regimes happen to correspond to height regimes for NGC 891 given
%% its edge-on orientation.

%% }

While we use $\tau_L$ as a proxy for star formation history we caution
that any single metric of age (including $\tau_L$) requires
assumptions about the underlying star formation history that will
introduce systematic offsets into the results. We quantify the
magnitude of this systematic uncertainty and find it to be, in a worst
case scenario, \asim 20\%. It is important to note, however, that for
studies of a single galaxy a worst-case scenario (which assigns a
completely random star formation history to each location in the
galaxy) is a large overestimation. On time scales greater than one
dynamical time we expect the the galaxy to be relatively well mixed
which causes the average SFH to be relatively constant and thus the
systematic uncertainties discussion in \S\ref{891_2:sec:sys_err} will
not apply. At large radii the dynamical timescale of NGC 891 is
roughly 1 Gyr, which means the systematic uncertainties in the I2 and
O DFK bins are greatly mitigated by mixing of stars within the galaxy.

%---------------
%% {\bf MAB: Again, you should note that the DFK parameterization gives a
%%   well defined framework for definning the age systematics that
%%   decouples the age-metallicity fitting degeneracies with the
%%   uncertainties inherent in age estimates from measurements of the
%%   integrated light of stellar populations (i.e., we don't measure
%%   individual stars so there is a limit to the available information in
%%   the spectra).

%% I know what you are getting at in the discussion of dynamical times
%% and mixing lengths because we talked about it, but we need to work a
%% bit more on our thinking and exposition. First, I am thinking that the
%% dynamical time, for a disk, works to sort out azimuthal variations,
%% i.e., azimuthal variations will tend to average out for stellar
%% populations older than that time-scale. I think we mean {\it radial}
%% mixing length, and this is the part that in my mind is fuzzy -- and
%% not just because of us. I don't think it is clear how much disks mix
%% radially, and if they do, how much the mixing depends on
%% height. Carlos and Elena claim that radial migration is most effective
%% for dynamically cold stars (small velocity dispersions and near the
%% mid-plane), so it is not clear how these migrating stars contribute to
%% the thicker disk component since nobody knows how to make a thick disk
%% without mergers or early-era turbulence (thick gas disk).

%% I think you are right that the 20\% systematic is an over-estimate,
%% but there could well be systematic trends in the SFH with radius and
%% height, and I don't think we can say more than that 20\% is an upper
%% limit and it is unlikely to apply at even the extrema in R and z in
%% any given galaxy.

%% }

%% ADE: Isn't that what I said? (I took out the mixing length
%% sentence, maybe this is a thought for another day).

The light weights shown in Figure \ref{891_2:fig:SFH_cuts} suggest an
even more optimistic picture, in which the SFH is relatively constant
across all of NGC 891. While the specific weights do change with $r$
and $z$, the general \emph{trends} for each DFK bin are the same at
all radii and heights. For example, the strength of the I2 population
increases with radius at all heights and the strength of the O
population decreases with radius at all heights.
%  It is more difficult
% to make similar statements about the Y and I1 stellar populations
% because they exist only at a narrow range of height and radius, but
% these populations make up a negligible contribution to the total
% systematic uncertainty. 
In other words the qualitative similarity in the light weights (a
proxy for the SFH) seen in Figure \ref{891_2:fig:SFH_cuts} indicates
that the underlying star formation rate does not vary \emph{in shape}
by large amounts across large regions of NGC 891. Thus uncertainties
that arise from assuming a star formation history (i.e.,
\S\ref{891_2:sec:sys_err}) only effect systematics on a galactic
scale; within NGC 891 these systematics contribute negligibly to the
total uncertainty.

%---------------
%% {\bf MAB: This is important: I need to see the SFR as requested for
%%   alternative versions of Figure \ref{891_2:fig:SFH_cuts}, i.e., once with
%%   mass-weights and then one with mass-weights divided by the time
%%   interval (SFR). That said (and with these in hand), I think you
%%   might have a good argument to make. Actually I think the point that
%%   is likely robust is not that the SFH is relatively constant with R
%%   and z, but that it changes smoothly at least in R. The SFR at recent
%%   times is quite different at different heights, particularly at $R<8$
%%   kpc. However, Y and I1 are not an issue for age systematics since
%%   these are dominated by the larger time bins of I2 and O.

%% Noted you have an incomplete sentence.}

Our 3-dimensional view of NGC 891 is enhanced by deprojecting the
observed, projected radii to cylindrical radii based on kinematic
measurements of stellar populations. Thus armed with fully 3D
information about the location of our stellar populations we examine
their distribution in ($r,|z|,\tau_L,Z_L,A_V$) space and find:

%+++++++++++++++
%% {\bf MAB: don't use ``true'' because there are uncertainties. You should
%% work these out (see earlier request) and summarize here.}

\begin{enumerate}

\item Confirmation of the results from Chapter
  \ref{chap:891_1}. Namely Young populations exist only below
  \val{0.4}{kpc}, age increases from 0.4 - \val{1}{kpc}, and saturates
  above \val{1}{kpc}.

\item The picture of inside-out galaxy formation seen in Chapter
  \ref{chap:891_1} is enhanced by the view that each generation of
  stars forms a flared disk, with later generations forming disks with
  larger radii. This view is similar to the prediction of
  \citet{Martig14a}; the galaxy disk is a superposition of flared
  disks that formed with increasing radii.

\item Two distinct epochs in the enrichment history of NGC 891. During
  the first few Gyr of its life NGC 891 existed in more of a
  closed-box state; from an initial seed of pristine gas each
  generation of stars formed from gas enriched by the generations
  before it at roughly a rate of \val{0.15}{\Zsol/Gyr}. However, about
  \val{6}{Gyr} ago new, metal-poor gas from outside the galaxy started
  to fall onto the disk of NGC 891. This new gas lowered the
  metallicity of stars formed in the last \val{6}{Gyr} and the
  indication is that metallicity will continue to decrease in the
  future. This rate of decrease is about \val{-0.18}{\Zsol/Gyr}.

\item Three groupings of stellar populations in the
  ($r,|z|,\tau_L,Z_L,A_V$) parameter space: a primary disk, a flared
  extension of the disk, and a metal-rich sequence at large radii and
  heights. These groupings are \emph{not} distinct features in NGC
  891. Instead they are the consequence of the formation and
  enrichment history present above. The primary disk is the main disk
  of the galaxy and is composed of multiple mono-age, flared disks
  that increase in radius with galactic age. The ``flare'' at large
  radii is corresponds to the most recent batch of star formation. The
  third sequence is the portion of the disk that formed during peak
  metallicity, \val{\asim 6}{Gyr} ago.

\end{enumerate}

%% \begin{enumerate}
  
%% \item A ``primary'' disk that exists at all heights (at least to
%%   \val{2.6}{kpc}) and radii less than \val{8}{kpc}. In this disk there
%%   is clear evidence for the presence of young stellar populations ($<
%%   \val{\asim 400}{Myr}$ ago) below \val{0.4}{kpc} and a lack of the
%%   same populations above \val{0.4}{kpc}, consistent with the results
%%   of \ref{chap:891_1}. Above this transition emission from
%%   the disk is dominated by I2 and O stars and the average population
%%   age increases with height. This disk also exhibits negative
%%   metallicity gradients with both radius and heights, which is
%%   consistent with observations of the Milky Way. It is likely that the
%%   primary ``disk'' is actually a superposition of multiple disk
%%   components like the ones found in \citet{Schechtman-Rook13,
%%     Schechtman-Rook14}.

%% \item A flared extension of the the primary disk at radii beyond
%%   \val{8}{kpc}. This flare has the same age and metallicity properties
%%   as the main disk, but with a scale height roughly twice that of the
%%   main disk (\val{\asim 0.9}{kpc}, compared to a V-band scale height
%%   measure at small radii by \citet{Xilouris99} of
%%   \val{0.4}{kpc}). This increase in scale height decreases the total
%%   surface density of dust at large radii and the extinction is
%%   correspondingly lower, especially near the midplane.
  
%% \item A sequence of intermediate-age, super-solar metallicity
%%   populations at large heights ($|z|> \val{\asim 0.9}{kpc}$) and radii
%%   ($r>\val{8}{kpc}$). Despite their old ages, the populations in this
%%   ``third sequence'' appear to come from a fundamentally different
%%   underlying distribution from stars at similar heights but smaller
%%   radii; within $r=\val{8}{kpc}$ light at large heights is dominated
%%   by the oldest stellar populations, but in the third sequence these
%%   heights are have strong contributions from intermediate-aged
%%   populations. Despite an overall higher metallicity this third
%%   sequence still shows internal metallicity gradients in $r$, and $z$
%%   consistent with the trends seen in the primary disk and flare. The
%%   origin of this population is not easily explained, but we suggest
%%   that it may be a signature of a flare that is older than the
%%   ``flare'' identified above. Simulations of Milky Way-like galaxies
%%   do suggest that early epochs of star formation can be highly flared
%%   via mergers and (to a lesser extent) radial migration
%%   \citep{Martig14b}, but this does not explain the high metallicity in
%%   this sequence. Any theories about sequence 3 will need to content
%%   with its curious combination of old ages and high metallicities that
%%   occur far from the center of the galaxy.

%% \end{enumerate}

%+++++++++++++++++
%% {\bf MAB: Good place to note connection to ASR's work. We also need to
%%   discuss the possible different picture that sequence 3 could be part
%%   of an older flare (a la Martig), or that the thicker disk is simply
%%   younger at larger radii, consistent with an inside-out scenario, but
%%   not necessarily one where flares occur. To be clear: the flare
%%   scenario has mono-age populations forming in a flared disk, where
%%   the flare gets larger with time. An alternate scenario is one also
%%   where the disk grows with time but the stars continue to form in a
%%   thin disk and are somehow heated. While the former seems to come
%%   more naturally out of current models the point is that we do not
%%   have an observational signature that would distinguish between the
%%   two. I think it is worth saying.}

%% This work represents an advance in our understand of the detailed
%% structure of NGC 891. In \ref{chap:891_1} we claimed that
%% vertical heating in NGC 891 appears have a nearly identical signature
%% as the Milky Way but with a radial and vertical metallicity gradient
%% the opposite of the Milky Way. With the methods outlined above we are
%% able to refine this picture; the ``normal'' primary disk of NGC 891
%% follows the trends, including metallicity, seen in the Milky Way, but
%% the overall picture is muddled by the presence of a intermediate-age,
%% super-solar metallicity sequence of populations at $r >$
%% \val{8}{kpc}. 

%----------------
%% {\bf MAB: To what extent do you think seq-3 is an extension of the
%%   flared thin disk? I would also make the broader point that we are
%%   setting the stage of studies of edge-on galaxies in integrated light
%%   to build a statistical picture of 3D stellar population gradients in
%%   massive spiral galaxies outside the MW. }

Despite the advances presented here there is ample opportunity for
future studies to improve upon our work and methods. In particular,
our treatment of systematic uncertainties is incomplete and we do not
address potential sources of systematics in our calculation of either
$A_V$ or $Z_L$. Our fit values of $A_V$ are dependent on the
extinction model assumed \citep[i.e.,][]{Charlot00}, but beyond that
we do not parametrize extinction with separate normalization
parameters for populations younger and older than \val{\asim 10}{Myr},
as recommended by \citet{Charlot00}. Furthermore we have not tested
how our results might change when using other common models
\citep[e.g.,][]{Calzetti94}.

Systematics in $Z_L$ likely arise from our simplistic treatment of how
populations with different metallicities mix together (i.e., we take a
straight average of $Z$). We also note that the specifics of chemical
evolution are often highly dependent on the SSP template library used
and we have only used one model \citepalias{Bruzual03} in this work.

%++++++++++++++
%% {\bf MAB: Can you be specific about you do not think is complete in
%%   our sys. uncertainty estimates for $A_V$ and $Z_L$? You say
%%   something about $A_V$ but is there anything else? What about $Z_L$?}

We also note that the use of diffusion k-means to down-select the SSP
library of \citet{Bruzual03} has not been rigorously tested for
effects of metallicity (\citetalias{Mosby15} only consider solar
metallicity SSPs). Furthermore, when the DFK spectra are constructed
from a weighted average of SSPs we assume a constant star formation
across each DFK bin, which very likely introduces systematics in the
DFK spectra and ultimately our derived parameters. Finally, the choice
of 4 DFK bins was primarily motivated by \citetalias{Mosby15}'s use of
very low S/N data. It is possible that using only 4 DFK bins
under-represents the full wealth of information contained in our
relatively high S/N data. Specifically, the current I2 DFK bin spans a
large range of stellar evolutionary time scales and could likely be
split into finer bins, which would allow one to leverage SSP models
that offer sophisticated treatment of the late phases of stellar
evolution \citepalias[e.g.,][]{Maraston11}

%+++++++++++++++++++
%% {\bf MAB: I would emphasize again the point about S/N and our
%%   relatively low time resolution in the 0.4-6 Gyr, which likely should
%%   be split and take advantage of testing alternative treatments of
%%   late-phases of stellar evolution (referencing Maraston).}

%% Things we didn't do:

%% o Systematics from DFK averaging
%% o Systematics on Z_L, A_V
%% o Multiple extinction normalizations for young, old a la Charlot & Fall
%% o Different DFK bins?

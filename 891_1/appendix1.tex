\begin{appendices}

\setcounter{table}{0}
\renewcommand{\thetable}{\Alph{section}\arabic{table}}
\setcounter{figure}{0}
\renewcommand{\thefigure}{\Alph{section}\arabic{figure}}

% \section*{Appendix A \\ \GP Performance Testing}
\section{\GP Performance Testing}
\label{sec:GPtesting}

\begin{deluxetable}{crrccccc}
\tablewidth{0.9\columnwidth}
\tablecaption{\GP Fiber Locations and Lab Data}
\tablehead{
    \colhead{Fiber} &
    \colhead{$\Delta\alpha$\tablenotemark{a}} &
    \colhead{$\Delta\delta$\tablenotemark{a}} &
    \colhead{diameter} &
    \colhead{$T_\mathrm{tot}$} &
    \colhead{$T_4$\tablenotemark{b}} &
    \colhead{$T_{4.4}$} &
    \colhead{$T_5$} \\
    \colhead{number} &
    \colhead{('')} &
    \colhead{('')} &
    \colhead{('')} &
    \colhead{} &
    \colhead{} &
    \colhead{} &
    \colhead{}
}
\startdata
1 &  64.86 &  34.02 &  1.87 &  0.84 &  0.67 &  0.58 &  0.49 \\
2 & -31.45 &  34.02 &  1.87 &  0.78 &  0.61 &  0.53 &  0.44 \\
3 &  32.52 & -51.50 &  1.87 &  0.75 &  0.51 &  0.44 &  0.36 \\
4 &  30.26 & -51.50 &  1.87 &  0.75 &  0.61 &  0.53 &  0.44 \\
5 &  28.00 & -51.50 &  1.87 &  0.76 &  0.59 &  0.52 &  0.43 \\
6 &  25.75 & -51.50 &  1.87 &  0.77 &  0.66 &  0.59 &  0.50
  %% 7 & 23.49 & -51.50 &  1.87\\
  %% 8 & 21.23 & -51.50 &  1.87\\
  %% 9 & 18.97 & -51.50 &  1.87\\
  %% 10 & 16.71 & -51.50 &  1.87\\
  %% 11 & 14.45 & -51.50 &  1.87\\
  %% 12 & 12.19 & -51.50 &  1.87\\
  %% 13 &  9.93 & -51.50 &  1.87\\
  %% 14 &  7.67 & -51.50 &  1.87\\
  %% 15 &  5.41 & -51.50 &  1.87\\
  %% 16 &  3.15 & -51.50 &  1.87\\
  %% 17 &  0.90 & -51.50 &  1.87\\
  %% 18 & -25.61 & 34.02 &  1.87\\
  %% 19 & 59.03 & 34.02 &  1.87\\
  %% 20 & 65.63 & 39.40 &  2.81\\
  %% 21 & 32.44 & -48.38 &  2.81\\
  %% 22 & 28.95 & -48.38 &  2.81\\
  %% 23 & 25.45 & -48.38 &  2.81\\
  %% 24 & 21.95 & -48.38 &  2.81\\
  %% 25 & 18.46 & -48.38 &  2.81\\
  %% 26 & 14.96 & -48.38 &  2.81\\
  %% 27 & 11.46 & -48.38 &  2.81\\
  %% 28 &  7.97 & -48.38 &  2.81\\
  %% 29 &  4.47 & -48.38 &  2.81\\
  %% 30 &  0.97 & -48.38 &  2.81\\
  %% 31 & -32.21 & 39.40 &  2.81\\
  %% 32 & 58.27 & 39.40 &  2.81\\
  %% 33 & 32.44 & -44.88 &  2.81\\
  %% 34 & 28.95 & -44.88 &  2.81\\
  %% 35 & 25.45 & -44.88 &  2.81\\
  %% 36 & 21.95 & -44.88 &  2.81\\
  %% 37 & 18.46 & -44.88 &  2.81\\
  %% 38 & 14.96 & -44.88 &  2.81\\
  %% 39 & 11.46 & -44.88 &  2.81\\
  %% 40 &  7.97 & -44.88 &  2.81\\
  %% 41 &  4.47 & -44.88 &  2.81\\
  %% 42 &  0.97 & -44.88 &  2.81\\
  %% 43 & -24.85 & 39.40 &  2.81\\
  %% 44 & 66.44 & 46.28 &  3.75\\
  %% 45 & 32.62 & -40.63 &  3.75\\
  %% 46 & 28.08 & -40.63 &  3.75\\
  %% 47 & 23.53 & -40.63 &  3.75\\
  %% 48 & 18.98 & -40.63 &  3.75\\
  %% 49 & 14.44 & -40.63 &  3.75\\
  %% 50 &  9.89 & -40.63 &  3.75\\
  %% 51 &  5.34 & -40.63 &  3.75\\
  %% 52 &  0.80 & -40.63 &  3.75\\
  %% 53 & -33.02 & 46.28 &  3.75\\
  %% 54 & 57.46 & 46.28 &  3.75\\
  %% 55 & 32.62 & -36.08 &  3.75\\
  %% 56 & 28.08 & -36.08 &  3.75\\
  %% 57 & 23.53 & -36.08 &  3.75\\
  %% 58 & 18.98 & -36.08 &  3.75\\
  %% 59 &  9.89 & -36.08 &  3.75\\
  %% 60 &  5.34 & -36.08 &  3.75\\
  %% 61 &  0.80 & -36.08 &  3.75\\
  %% 62 & -24.04 & 46.28 &  3.75\\
  %% 63 & 61.95 & 31.25 &  4.69\\
  %% 64 & 33.33 & -30.80 &  4.69\\
  %% 65 & 27.79 & -30.80 &  4.69\\
  %% 66 & 22.25 & -30.80 &  4.69\\
  %% 67 & 16.71 & -30.80 &  4.69\\
  %% 68 & 11.17 & -30.80 &  4.69\\
  %% 69 &  5.63 & -30.80 &  4.69\\
  %% 70 &  0.09 & -30.80 &  4.69\\
  %% 71 & -28.53 & 31.25 &  4.69\\
  %% 72 & 33.33 & -25.26 &  4.69\\
  %% 73 & 27.79 & -25.26 &  4.69\\
  %% 74 & 22.25 & -25.26 &  4.69\\
  %% 75 & 16.71 & -25.26 &  4.69\\
  %% 76 & 11.17 & -25.26 &  4.69\\
  %% 77 &  5.63 & -25.26 &  4.69\\
  %% 78 &  0.09 & -25.26 &  4.69\\
  %% 79 & 61.95 & 36.79 &  4.69\\
  %% 80 & 33.33 & -19.72 &  4.69\\
  %% 81 & 27.79 & -19.72 &  4.69\\
  %% 82 & 22.25 & -19.72 &  4.69\\
  %% 83 & 16.71 & -19.72 &  4.69\\
  %% 84 & 11.17 & -19.72 &  4.69\\
  %% 85 &  5.63 & -19.72 &  4.69\\
  %% 86 &  0.09 & -19.72 &  4.69\\
  %% 87 & -28.53 & 36.79 &  4.69\\
  %% 88 & 61.95 & 42.92 &  5.62\\
  %% 89 & 33.42 & -13.37 &  5.62\\
  %% 90 & 26.73 & -13.37 &  5.62\\
  %% 91 & 20.05 & -13.37 &  5.62\\
  %% 92 & 13.37 & -13.37 &  5.62\\
  %% 93 &  6.68 & -13.37 &  5.62\\
  %% 94 &  0.00 & -13.37 &  5.62\\
  %% 95 & -28.53 & 42.92 &  5.62\\
  %% 96 & 33.42 & -6.68 &  5.62\\
  %% 97 & 26.73 & -6.68 &  5.62\\
  %% 98 & 20.05 & -6.68 &  5.62\\
  %% 99 & 13.37 & -6.68 &  5.62\\
  %% 100 &  6.68 & -6.68 &  5.62\\
  %% 101 &  0.00 & -6.68 &  5.62\\
  %% 102 & 61.95 & 49.64 &  5.62\\
  %% 103 & 33.42 &  0.00 &  5.62\\
  %% 104 & 26.73 &  0.00 &  5.62\\
  %% 105 &  0.00 &  0.00 &  5.62\\
  %% 106 & 13.37 &  0.00 &  5.62\\
  %% 107 &  6.68 &  0.00 &  5.62\\
  %% 108 & 20.05 &  0.00 &  5.62\\
  %% 109 & -28.53 & 49.64 &  5.62\\
\enddata
\label{tab:GP_cal}
\tablenotetext{a}{Distance from fiber 105.}
\tablenotetext{b}{An estimate of on-bench performance. See Equation \ref{eq:T_FRD}.}
\tablecomments{Table \ref{tab:GP_cal} is published in its entirety in
  the machine-readable format. A portion is shown here for guidance
  regarding its form and content.}
\end{deluxetable}


Before installation all of the \GP fibers were tested for throughput
performance using the Wisconsin Test Stand
\citep{Bershady04,Crause08,Eigenbrot12}. This stand is a
double-differential imaging comparator. It consists of an input stage
that reimages an illuminated aperture through a controllable aperture
at an intermediate pupil, and an output stage that consists of a
collimator that places the output pupil from the reimager or fiber
output onto a CCD detector. The experiment consists of measuring the
input beam differentially between a straight-through configuration and
the collimated fiber output. During the entire process the stability
of the filtered input beam is monitored with a photodiode.

Tests were performed in the Johnson $V$ band with an input beam set to
match the WIYN input beam of $f$/6.3 without the 17\% central
obstruction of the telescope. The total throughput ($T_{\rm tot}$) is
defined as all of the fiber-output light captured by the CCD (roughly
corresponding to $f$/2.2, compared with the fiber numerical aperture
of $f$/2.3)
%From 54mm L3 / 1024px * 0.024mm/px. NA = 0.22 and N = (2*arctan(NA))^-1
divided by the all of the light from the input beam captured by the
same CCD. This gives a good indication of the total transmission
through the fiber, but ignores the effects of FRD on the delivered
throughput on the Bench Spectrograph due to the optical stops
therein. FRD describes the tendency for fibers to increase the entropy
in an optical beam; light injected into a fiber at a particular
$f$-ratio emerges at a smaller (faster) $f$-ratio \citep{Angel77}.

The primary impact of FRD is from light loss from obstructions inside
the Bench Spectrograph as detailed in \cite{Bershady04} and
\cite{Bershady05}.  With the upgrade \citep{Bershady08} the low-order
gratings and camera objective are sufficiently near the pupil and are
of sufficient size that the limiting stop is from the collimator. The
camera objective does begin to vignette for off-axis fields point in
wavelength and pseudo-slit position, but for operational purposes we
consider the on-axis field point for definning throughput losses due
to FRD.  The collimator accepts light up to $f$/4 with only minimal
obstruction, and is completely unobstructed at $f$/4.4, while the
optical design (in terms of aberrations) is optimized for f/5.

To characterize the impact of FRD on the the delivered throughput we
define the quantity
\begin{equation}
\label{eq:T_FRD}
  T_{\mathrm{4}} = \frac{F_{\mathrm{out}}(f<f/4)}{F_{\mathrm{in}}(f<f/6.3)},
\end{equation}
where $F_\mathrm{in}$ and $F_\mathrm{out}$ represent flux from the the
input and fiber output beams, respectively. $T_4$ is essentially a
throughput measurement that accounts for the impact of FRD, and should
be representative of how the fibers will perform as part of the
WIYN/Bench system.  We compute comparable quantities for output
f-ratios of f/4.5 and f/5.  The throughput in each of these apertures,
as well as the total throughput are included in Table \ref{tab:GP_cal}
as $T_{\rm tot}$, $T_4$, $T_{4.4}$, $T_{5}$.

\begin{figure*}[htb]
  \centering
  \includegraphics[width=0.35\textwidth]{891_1/figs/gradpak_map.pdf}
  \includegraphics[width=0.35\textwidth]{891_1/figs/gradpak_L_map.pdf}
\vskip -0.25in
  \caption{\label{fig:TL_FRD} Maps of laboratory measurements of the
    \GP IFU performance. Throughput ($T_{\mathrm{FRD}}$, Equation
    \ref{eq:T_FRD}) is shown in the left-hand panel, while throughput
    losses ($L_{\mathrm{FRD}}$, Equation \ref{eq:L_FRD}), is shown in
    the right-hand panel. Performance values are given in the color
    scale at the top of each panel.}
\end{figure*}

\begin{figure}
  \centering
  \includegraphics[width=0.4\textwidth]{891_1/figs/gradpak_facefig.pdf}
  \caption{\label{fig:gradpak_face}Detail of \GP fiber face after
    polishing. The \val{25.4}{\mu m} (0.24'') shims between each block
    fiber-size group are visible in the central array. The hole in the
    5th fiber row from the bottom is a fiber that was broken during
    polishing. The two sky-fiber groups, not shown to scale, have
    locations marked in Figure \ref{fig:GradPak}.  The \val{200}{\mu
      m} (1.87'') sky fibers are not visible in this image due to
    imperfect illumination conditions at the slit end when this
    picture was taken.}
\end{figure}

Figure \ref{fig:TL_FRD} contains throughput measurements for all of
the active \GP fibers. Throughput losses caused by FRD are spatially
coherent and are larger on the top and left (North and East) side of
the IFU. This is likely due to surface scattering at the IFU face
\citep{Eigenbrot12} caused by an uneven polish. Figure
\ref{fig:gradpak_face} shows evidence of this variable polish in the
two sky fiber groups; the SW sky group show significantly worse polish
than the NE group and has a correspondingly lower throughput. It is
important to note that the FRD losses do \emph{not} appear to be
caused by edge fibers pushing against the aluminum structure, as
evidenced by the relatively high throughput seen on the left side of
the IFU. This is consistent with previous studies \citep{Bershady04}
that suspected removal from an IFU molding fixture to be the primary
cause of stress-induced FRD. Variations in polish quality appears to
be caused by detritus from the aluminum fixture.

As a check on the lab measurements of $T_4$ we also compare total
fiber transmission recorded during the observing program described in
\S\ref{sec:obs}. For these measurements a stiched dome flat (see
\S\ref{sec:flats} was used as a good approximation of a uniform
illumination at the fiber input. The total light transmitted by each
fiber is computed by adding together all wavelength channels for each
fiber after the data were spectrally extracted. Figure
\ref{fig:count_tput} shows are comparison between lab ($T_4$) and
on-telescope performance. ``Counts'' in this figure are from
dome-flats combined as described in \S\ref{sec:flats}), and are the
sum across all wavelengths of the extracted fiber traces.  As expected
from Figure \ref{fig:TL_FRD} the highest throughputs are found in the
middle of the array, with a gradual drop off in performance towards
the end of the slit. The right panel of Figure \ref{fig:count_tput}
shows a tight correlation between our lab measurements and the on-sky
performance. This shows that stresses during installation and the
performance of the Bench Spectrograph only cause a $\pm$5\% rms
modulation in the throughput compared to what was measured in the lab.

We also measure the magnitude of FRD experienced by each
fiber. Because FRD represents a scattering of input light to larger
output angles a comparison between throughput at two different output
$f$-ratios gives an approximation of the severity of FRD in each
fiber. We define
\begin{equation}
\label{eq:L_FRD}
  L_\mathrm{FRD} = 1 - \frac{T_5}{T_4},
\end{equation}
which quantifies the amount of light scattered to smaller $f$-ratios
(larger angles) than $f/5$ as a measure of the severity of FRD
experienced by each fiber.

\begin{figure*}
  \centering
  \includegraphics[width=0.9\textwidth]{891_1/figs/gradpak_count_plots.pdf}
  \caption{\label{fig:count_tput} Left: Relative fiber transmission
    measured \emph{in-situ} on the WIYN Bench Spectrograph. Vertical
    lines demark transition between different fiber sizes, as
    labeled. Right: Comparison between the \emph{in-situ} performance
    and $T_4$, as measured in the lab after construction. The dashed
    line represents a linear regression to the data with given
    correlation coefficient and scatter. Points are color-coded by
    fiber number (slit position), and sky fibers are marked with black
    squares.}
\end{figure*}

\begin{figure*}
  \centering
  \includegraphics[width=0.9\textwidth]{891_1/figs/gradpak_Lplots.pdf}
  \caption{\label{fig:FRD_loss} Performance metrics for
    \GP. $L_\mathrm{FRD}$ (Equation \ref{eq:L_FRD}) as a function of
    slit location (right) and $T_4$ (transmission through an f/4
    aperture). Points are color-coded by fiber number (slit position),
    and sky fibers are marked with black squares.}
\end{figure*}

Table \ref{tab:GP_cal} contains measurements and Figure
\ref{fig:TL_FRD} shows a map of $L_\mathrm{FRD}$ for each \GP
fiber. Figure \ref{fig:FRD_loss} compares this quantity with $T_4$. We
find that fibers with low $T_4$ also tend to have high FRD losses
($L_\mathrm{FRD}$), which indicates that FRD is a significant source
of throughput loss in the \GP and Bench Spectrogtraph system.  However
this is relatively more pronounced for smaller fiber sizes, as seen in
the bifrucation in the right-hand panel of Figure
\ref{fig:FRD_loss}. Smaller fibers tend to suffer from larger amounts
of FRD, which may have been caused in \GP due to handling-induced
stresses; these fibers were more likely to bend and tangle during
construction.  The corollary is that lower throughput in larger fibers
isn't always the result of increased FRD.

A complete set of measurements showing FRD losses as a function of
output $f$-ratio for each fiber can be found in the supplemental
materials online. {\bf [can they? I'm thinking this would basically be
    a copy of what is available at
    \url{www.astro.wisc.edu/~eigenbrot/PAK} MAB: include material here
    since the www at the above address will not be archival.]}.

% \begin{deluxetable}{ccccc}
    \tablewidth{0.9\columnwidth}
\tablecaption{\GP Lab Data}
\tablehead{
    \colhead{Fiber} &
    \colhead{$T_\mathrm{tot}$} &
    \colhead{$T_4$\tablenotemark{a}} &
    \colhead{$T_{4.4}$} &
    \colhead{$T_5$} \\
    \colhead{Number} &
    \colhead{} &
    \colhead{} &
    \colhead{} &
    \colhead{}
}
\startdata
   1 &  0.84 &  0.67 &  0.58 &  0.49 \\
   2 &  0.78 &  0.61 &  0.53 &  0.44 \\
   3 &  0.75 &  0.51 &  0.44 &  0.36 \\
   4 &  0.75 &  0.61 &  0.53 &  0.44 \\
   5 &  0.76 &  0.59 &  0.52 &  0.43 \\
   6 &  0.77 &  0.66 &  0.59 &  0.50 \\
  %%    7 &  0.79 &  0.67 &  0.59 &  0.50 \\
  %%    8 &  0.79 &  0.71 &  0.64 &  0.55 \\
  %%    9 &  0.77 &  0.66 &  0.59 &  0.49 \\
  %%   10 &  0.78 &  0.69 &  0.62 &  0.52 \\
  %%   11 &  0.80 &  0.72 &  0.64 &  0.53 \\
  %%   12 &  0.77 &  0.69 &  0.61 &  0.51 \\
  %%   13 &  0.82 &  0.76 &  0.69 &  0.59 \\
  %%   14 &  0.80 &  0.77 &  0.71 &  0.62 \\
  %%   15 &  0.81 &  0.80 &  0.78 &  0.71 \\
  %%   16 &  0.80 &  0.79 &  0.76 &  0.70 \\
  %%   17 &  0.80 &  0.78 &  0.74 &  0.66 \\
  %%   18 &  0.73 &  0.66 &  0.64 &  0.58 \\
  %%   19 &  0.86 &  0.89 &  0.88 &  0.85 \\
  %%   20 &  0.85 &  0.87 &  0.86 &  0.80 \\
  %%   21 &  0.76 &  0.69 &  0.62 &  0.52 \\
  %%   22 &  0.81 &  0.76 &  0.69 &  0.59 \\
  %%   23 &  0.82 &  0.78 &  0.72 &  0.62 \\
  %%   24 &  0.84 &  0.82 &  0.77 &  0.67 \\
  %%   25 &  0.83 &  0.79 &  0.72 &  0.61 \\
  %%   26 &  0.83 &  0.83 &  0.79 &  0.68 \\
  %%   27 &  0.83 &  0.80 &  0.75 &  0.66 \\
  %%   28 &  0.85 &  0.83 &  0.78 &  0.67 \\
  %%   29 &  0.85 &  0.84 &  0.79 &  0.69 \\
  %%   30 &  0.85 &  0.80 &  0.72 &  0.60 \\
  %%   31 &  0.80 &  0.74 &  0.69 &  0.60 \\
  %%   32 &  0.86 &  0.89 &  0.88 &  0.84 \\
  %%   33 &  0.78 &  0.72 &  0.65 &  0.56 \\
  %%   34 &  0.82 &  0.80 &  0.74 &  0.65 \\
  %%   35 &  0.84 &  0.84 &  0.80 &  0.73 \\
  %%   36 &  0.83 &  0.81 &  0.76 &  0.67 \\
  %%   37 &  0.84 &  0.84 &  0.79 &  0.70 \\
  %%   38 &  0.85 &  0.84 &  0.78 &  0.67 \\
  %%   39 &  0.85 &  0.85 &  0.80 &  0.69 \\
  %%   40 &  0.84 &  0.85 &  0.82 &  0.74 \\
  %%   41 &  0.85 &  0.83 &  0.76 &  0.65 \\
  %%   42 &  0.86 &  0.85 &  0.79 &  0.67 \\
  %%   43 &  0.79 &  0.77 &  0.75 &  0.70 \\
  %%   44 &  0.85 &  0.88 &  0.86 &  0.79 \\
  %%   45 &  0.77 &  0.75 &  0.71 &  0.63 \\
  %%   46 &  0.81 &  0.78 &  0.74 &  0.66 \\
  %%   47 &  0.85 &  0.85 &  0.82 &  0.73 \\
  %%   48 &  0.85 &  0.83 &  0.77 &  0.67 \\
  %%   49 &  0.81 &  0.80 &  0.77 &  0.68 \\
  %%   50 &  0.85 &  0.88 &  0.86 &  0.78 \\
  %%   51 &  0.85 &  0.88 &  0.85 &  0.78 \\
  %%   52 &  0.85 &  0.88 &  0.85 &  0.77 \\
  %%   53 &  0.81 &  0.79 &  0.76 &  0.69 \\
  %%   54 &  0.86 &  0.89 &  0.88 &  0.83 \\
  %%   55 &  0.77 &  0.76 &  0.73 &  0.67 \\
  %%   56 &  0.83 &  0.83 &  0.78 &  0.68 \\
  %%   57 &  0.83 &  0.82 &  0.79 &  0.73 \\
  %%   58 &  0.84 &  0.86 &  0.85 &  0.81 \\
  %%   59 &  0.85 &  0.88 &  0.87 &  0.81 \\
  %%   60 &  0.85 &  0.88 &  0.87 &  0.83 \\
  %%   61 &  0.85 &  0.88 &  0.87 &  0.83 \\
  %%   62 &  0.79 &  0.78 &  0.75 &  0.67 \\
  %%   63 &  0.87 &  0.88 &  0.85 &  0.78 \\
  %%   64 &  0.76 &  0.72 &  0.69 &  0.63 \\
  %%   65 &  0.83 &  0.84 &  0.82 &  0.77 \\
  %%   66 &  0.84 &  0.85 &  0.83 &  0.78 \\
  %%   67 &  0.86 &  0.87 &  0.85 &  0.78 \\
  %%   68 &  0.86 &  0.87 &  0.85 &  0.80 \\
  %%   69 &  0.86 &  0.88 &  0.85 &  0.79 \\
  %%   70 &  0.87 &  0.89 &  0.86 &  0.80 \\
  %%   71 &  0.78 &  0.73 &  0.70 &  0.64 \\
  %%   72 &  0.75 &  0.69 &  0.66 &  0.60 \\
  %%   73 &  0.81 &  0.81 &  0.79 &  0.73 \\
  %%   74 &  0.85 &  0.87 &  0.86 &  0.80 \\
  %%   75 &  0.85 &  0.86 &  0.84 &  0.79 \\
  %%   76 &  0.86 &  0.88 &  0.86 &  0.82 \\
  %%   77 &  0.86 &  0.88 &  0.86 &  0.82 \\
  %%   78 &  0.86 &  0.87 &  0.85 &  0.79 \\
  %%   79 &  0.86 &  0.89 &  0.87 &  0.82 \\
  %%   80 &  0.76 &  0.73 &  0.69 &  0.63 \\
  %%   81 &  0.82 &  0.82 &  0.79 &  0.73 \\
  %%   82 &  0.84 &  0.84 &  0.82 &  0.78 \\
  %%   83 &  0.83 &  0.84 &  0.82 &  0.78 \\
  %%   84 &  0.85 &  0.87 &  0.85 &  0.80 \\
  %%   85 &  0.86 &  0.87 &  0.86 &  0.81 \\
  %%   86 &  0.85 &  0.86 &  0.83 &  0.75 \\
  %%   87 &  0.81 &  0.78 &  0.75 &  0.69 \\
  %%   88 &  0.87 &  0.86 &  0.83 &  0.76 \\
  %%   89 &  0.77 &  0.71 &  0.67 &  0.58 \\
  %%   90 &  0.80 &  0.77 &  0.74 &  0.66 \\
  %%   91 &  0.84 &  0.84 &  0.81 &  0.75 \\
  %%   92 &  0.83 &  0.82 &  0.79 &  0.74 \\
  %%   93 &  0.83 &  0.83 &  0.81 &  0.76 \\
  %%   94 &  0.86 &  0.86 &  0.84 &  0.79 \\
  %%   95 &  0.79 &  0.74 &  0.71 &  0.65 \\
  %%   96 &  0.76 &  0.70 &  0.66 &  0.57 \\
  %%   97 &  0.79 &  0.75 &  0.71 &  0.62 \\
  %%   98 &  0.84 &  0.84 &  0.82 &  0.78 \\
  %%   99 &  0.85 &  0.86 &  0.84 &  0.79 \\
  %%  100 &  0.82 &  0.81 &  0.79 &  0.73 \\
  %%  101 &  0.82 &  0.81 &  0.77 &  0.70 \\
  %%  102 &  0.86 &  0.84 &  0.80 &  0.71 \\
  %%  103 &  0.76 &  0.71 &  0.67 &  0.61 \\
  %%  104 &  0.79 &  0.77 &  0.74 &  0.68 \\
  %%  105 &  0.75 &  0.70 &  0.66 &  0.60 \\
  %%  106 &  0.78 &  0.73 &  0.69 &  0.63 \\
  %%  107 &  0.80 &  0.77 &  0.74 &  0.68 \\
  %%  108 &  0.81 &  0.80 &  0.76 &  0.70 \\
  %%  109 &  0.81 &  0.79 &  0.75 &  0.69 \\
\enddata
\label{tab:GP_lab}
\tablenotetext{a}{An estimate of on-bench performance. See Equation \ref{eq:T_FRD}.}
\tablecomments{Table \ref{tab:GP_lab} is published in its entirety in the machine-readable format. A portion is shown here for guidance regarding its form and content.}
\end{deluxetable}


\section{Grating Optimization}
\label{sec:grating}

\begin{figure*}[htb]
\centering
\vskip -1.25in
  \includegraphics[width=0.9\textwidth]{891_1/figs/blaze_comp_land.pdf}
\vskip -1.25in
  \caption{\label{fig:grating_comp} Efficiency comparison between
    400@4.2 and 600@10.1 gratings based on dome-flat exposures using
    the same fibers, spectrograph camera-collimator angle, and lamp
    temperature and intensity. Details are provided in text. The top
    panel shows counts for each grating, while the bottom panel shows
    their ratio and the prediction (line) based on the blaze functions
    for idealized gratings.}
\end{figure*}

% inferring camera fl. of 277.1mm,
% 400@4.2 with alpha=21.8 covers 3597-7867 A; blaze wave is 3496 A.
% 400@4.2 with alpha=21.53 covers 3372-7640 A; blaze wave is 3496 A.
% 600@10.1  with alpha=24.33 covers 3795-6657; blaze wave is 5666 A.

% Notes: effective camera fl at central waves of 5524 (g600) and 5733
% (g400) is 277.1 mm compared to nominal value of 285 mm. This is due to
% chromatic behvior of all-refractive camera and the ned to
% significantly modify focus for good image-quality across our primary
% bandpass.

% The band-pass in the plots can be shifted redward or bluerward with a
% commensurate shift (or offset) in the camera-collimator angle
% w.r.t. what is plotted here.

We chose the 400@4.2 grating out of the library of gratings available
on the WIYN Bench Spectrograph because it allows us to capture
simultaneously spectra from Ca H\&K to \Ha while still maintaining
relatively high efficiency at the blue end of the spectrum.  To
optimize our grating choice we compared the 400@4.2 grating to the
600@10.1 grating. The latter provides a narrower wavelength range but
at higher resolution, and, most importantly, the covered range is
marginally sufficient to meet our scientific objectives.  The 600@10.1
grating is often the ``go to'' grating for low-resolution programs
centered around \val{5500}{\AA}, particularly because it is the newest
reflection grating and purportedly has the highest diffraction
efficiency.

For testing during engineering time we compared the 400@4.2 grating
set to 21.8$^{\circ}$ and the 600@10.1 grating set to 24.33$^{\circ}$,
both for a fixed camera-collimator angle of 30$^{\circ}$ (this is the
nominal configuration for low-order gratings).\footnote{At these
  wavelengths the effective camera focal-length for this
  all-refractive compound optic is 277.1 mm not the nominal 285 mm
  quoted in reference manuals.} These grating incidence angles gave
wavelength ranges of \val{3600}{\AA}$< \lambda <$ \val{7867}{\AA} and
\val{3794}{\AA}$< \lambda <$ \val{6655}{\AA}, respectively. The blue
Hydra cable (300 $\mu$m core fibers) was used to observe dome flats
illuminated with identical lamp intensity (and temperature)
settings. The dome lamps are known to be stable to better than 10\%
over a broad wavelength range. Although we were concerned that
temperature instability might be a factor at the blue end of our
spectral range, our results are consistent with a highly stable
illumination over our full spectral range.

The fiber flux in the dome-flat spectra was measured on the raw
two-dimensional images before extraction. This was done in order to
exclude defocus effects from the fractional flux extracted along the
trace in wavelength, and between the two grating configurations for
which the detailed focus changes with wavelength are different.
Identical fibers and extraction regions were measured in both
configurations. The regions account for a small lateral shift along
the slit due to a slight difference in the alignment angle between
gratings orthogonal to the dispersion axis (this variance is just a
mechanical tolerance in the grating mount). Taking advantage of the
numerous broken Hydra fibers, we identified well-separated
transmitting fibers that had ample separation for extracting ``on''
and ``off'' signal regions (or aperures) at all wavelengths.  These
apertures were 10-11 pixels wide, and were adjacent on the CCD.  The
resulting counts were differenced and then scaled for the
corresponding exposure time and linear dispersion. 

Figure \ref{fig:grating_comp} shows the result of this experiment. It
is evident from the top panel that the 400@4.2 has greater efficiency
compared to the 600@10.1 grating blueward of 4700\AA. Since the
spectrograph configurations were identical except for the gratings,
under the assumption that the dome-flat illumination was constant, the
ratio of the two curves is equivalent to the ratio of the grating
blaze functions. As the bottom panel shows, this is very close to what
would be expected from simply computing the theoretical blaze
functions for idealized versions of these two gratings. If anything,
the 400@4.2 apepars to have 10\% higher efficiency across all
wavelengths than the idealized case.

\begin{figure*}[htb]
  \centering
\vskip -1.25in
  \includegraphics[width=0.9\textwidth]{891_1/figs/blaze_plot_land.pdf}
\vskip -1.25in
  \caption{\label{fig:spec_config} Wavelength blaze and coverage for
    the Bench Spectrograph and the 400 l/mm grating blazed at 4.2 deg
    (top panel) and the 600 l/mm grating blazed at 10.1 deg (bottom
    panel) as a function of grating incidence angle and
    camera-collimator angle with the central wavelength held
    constant. Upper and lower wavelength limits and the blaze peak are
    marked with the red, blue, and thick-dashed lines,
    respectively. The minimal desired wavelength range between the
    most blue-shifted Ca-K line and most redshifted H$\alpha$ lines
    (3937-6480\AA; 300-800 km/s) is shown as the grey shaded region
    The camera-collimator angle is marked with the thin dotted line,
    and corresponds to the right-hand vertical scale; the nominal
    value of 30 deg is marked with a dot.}
\end{figure*}

Since we are working with a bench-mounted spectrograph that in
principle is highly configurable, we also considered if additional
spectrograph layout modification might further optimize
performance. In particular, by altering both the grating (incidence)
angle and the camera-collimator angle, it is possible to change the
blaze wavelength while keeping the wavelength coverage on the detector
roughly constant. The concept is illustrated in \ref{fig:spec_config}.
The effect of increasing the camera-collimator angle for low-blaze
gratings shifts the blaze wavelength to the blue. This is not
desirable for the 400@4.2 grating which already has a very blue blaze
wavelength at the nominal camera-collimator angle of 30$^{\circ}$, but
this is potentially relevant for the 600@10.1 grating. Unfortunately,
significant shifts of the blaze wavelength require very large
camera-collimator angles that are both geometrically impractical on
the existing optical bench and lead to an unacceptable decrease in
spectral coverage. The excercise does conclude that a larger
camera-collimator angle of roughly 40$^{\circ}$ would be preferable
for the 600@10.1 grating, but its performance in the far blue would
still fall short of the 400@4.2 grating.

\section{$\tau$ Model of Star Formation}
\label{sec:tau_model}
Our galaxy models are constructed from the simple stellar population
(SSP) library of \citet{Bruzual03} with an initial mass function from
\citet{Chabrier03}. These SSPs rely on the STELIB stellar library
\citep{LeBorgne03}. The goal of any regularized galaxy model is to
assign individual SSP weights based on a modeled star formation
history (SFH). Our SSP basis set is normalized to \val{1}{M_{\odot}}
and thus the SSP weights will be the total mass of stars in a give
population. We let the mass of a model galaxy at lookback time, $t$,
be
\begin{equation}
\label{eq:M(t)}
M(t) = \int_{t_{\rm form}}^t \psi(t') dt',
\end{equation}
where $\psi(t)$ is the amount of mass in stars produced at loockback time
$t$. Equation \ref{eq:M(t)} is generally true for any star formation
history. For a $\tau$ model we use
\begin{equation}
\label{eq:taumodel}
\psi(t) = \psi_0 e^{-t/\tau_{SF}}.
\end{equation}

Taking the above, our galaxy flux, $G(\lambda)$, is constructed by weighting
individual SSPs, $f_i(\lambda)$, by the mass formed during their formation age:
\begin{equation}
G(\lambda) = \sum_i^N f_i(\lambda) M_i,
\end{equation}
where
\begin{equation}
M_i = \int_{t_2}^{t_1} \psi(t') dt'.
\end{equation}

In practice our SSP ages are discreet and we need to choose a range of time
($t_2 - t_1$) over which SSP contributes to star formation. We set these
limits of integration such that an equal amount of mass is formed in each half
of a logarithmically-spaced SSP age bin. In other words, given an SSP of age
$t_i$, the limits of integration are
\begin{equation}
t_{2,i} = \frac{\log (t_{i+1}) - \log (t_i)}{2} = t_{1,i+1}.
\end{equation}
We set the end points $t_{1,0} = 0$ and $t_{2,N} = t_{\rm form}$.

In the situation where $t_{\rm form}$ is less than the oldest SSP in
our library the SSP mass bins will be unevenly spaced in age. To fix
this problem we assign each SSP an age that is weighted by the amount
of mass formed during the time assigned to its bin. In other words,
the weighted age of an SSP with age $t_i$ is the midpoint of mass bin
$M_i$. This weighted age, $t_{i,w}$, is defined such that
\begin{equation}
\int_{t_1}^{t_{i,w}} \psi(t') dt' = \int_{t_{i,w}}^{t_2} \psi(t') dt'.
\end{equation}
If we assume the $\tau$ model described in Eq. \ref{eq:taumodel} the weighted age is
\begin{equation}
t_{i,w} = \tau_{SF} \log\left( 0.5 \left( e^{t_1/\tau_{SF}} + e^{t_2/\tau_{SF}} \right)\right).
\end{equation}

Finally, we adopt the extinction law of \citet{Charlot00} with a value
of $A_V=1.63$, which defines a reddening term that is constant across all SSPs, 
\begin{equation}
R(\lambda) = e^{-\frac{A_V}{1.086} \left(\lambda/\val{5500}{\AA}\right)^{-0.7}},
\end{equation}
 so our final galaxy is given by
\begin{equation}
G(\lambda) = R(\lambda)\sum_i^N f_i(\lambda) M_i.
\end{equation}

As a proxy for star formation history we define the mean light-weighted age as
\begin{equation}
\label{eq:MLWA}
\tau_L = \frac{\sum_k\left[S(\lambda_K) R(\lambda_k) \sum_i f_i(\lambda_k) M_i t_i\right]}{\sum_{k}\left[S(\lambda_k) R(\lambda_k) \sum_i f_i(\lambda_k) M_i\right]}
%% \tau_L = \frac{\sum_k S(\lambda_k)\left(\frac{\sum_{i,j} \psi(t_i,Z_j) f(\lambda, t_i,Z_j) t_i}{\sum_{i,j} \psi(t_i,Z_j) f(\lambda, t_i, Z_j)} \right)}{\sum_k S(\lambda_k)},
\end{equation}
where $S(\lambda_j)$ defines the bandpass over which the age is computed. We
set $S(\lambda_j)$ to be flat over \val{5450}{\AA} $\leq\lambda\leq$
\val{5550}{\AA} and zero everywhere else.

\end{appendices}

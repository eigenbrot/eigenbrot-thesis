\section{Summary}
\label{891_1:sec:summary}

We set out to measure vertical population gradients in NGC 891,
motivated to determine if the long-known vertical age gradients in our
solar neighborhood are seen in another spiral galaxy with comparable
rotation speed and morphology as the Milky Way.  The advent of large
spectroscopic surveys of Milky Way stars (see the Introduction) has
renewed interest in the age and metallicity gradients seen in the
Milky Way. Studies using these surveys
\citep[e.g.,][]{Bovy12c,Hayden14,Hayden15} are establishing the
relationships between age, velocity (height), metallicity and
compositional abundance of stars in unprecedented detail and accuracy,
but only for one galaxy.  Recent measurements have established the
age-velocity-metallicity relationship in M31's disk
\citep{Dorman15}. In this context we have undertaken to make the first
comprehensive determination of both vertical and radial population
gradients in integrated light for a massive spiral galaxy outside of
the Local Group.

To aid detailed studies of nearby galaxies like NGC 891 we constructed
a unique fiber integral field unit for the WIYN telescope's Bench
Spectrograph called \GP. \GP is part of a pair of variable-pitch IFUs
that share a common cable and spectrograph mount; the other IFU is
called HexPak, described elsewhere.  The \GP instrument is optimized
to measure vertical gradients in edge-on galaxies like NGC 891. In
this paper we have detailed the primary attributes of this instrument
relevant for astronomical observations, and we have presented
laboratory and on-telescope calibration of its performance.  The
unique design of this instrument required novel construction methods
that had some impact on its performance. In particular, the choice of
an aluminum fixture for \GP resulted in a sub-optimal polish for some
fibers that induced throughput losses (Appendix \ref{chap:GPtesting}),
however, the over-all performance of the IFU is quite high with a mean
throughput of 80\%.

The multi-pitch nature of \GP also required modifications to the
standard IFU data reduction routines, specifically flat-field and
absolute flux calibrations. We have shown (\S\ref{891_1:sec:data_reduction})
that these modifications do not significantly affect the quality of
the final data product. In fact, in the absence of atmospheric
dispersion correction, the presence of very large (6'' diameter)
fibers makes it possible to flux-calibrate \GP data very accurately in
both a relative and absolute sense ($\sim$5\%), as we have
demonstrated.

% Despite the learning curve associated with using a new type of
% instrument (here a variable pitch IFU), and the limitations of an
% outdated spectrograph, 

We have obtained spectroscopic observations of NGC 891 covering the
wavelength range from 337 to 764 nm, with high-quality data in the
range from 380 to 670 nm. Spectral resolution varies with wavelength
and fiber size in the range from 100 to 500 km s$^{-1}$; at 520 nm the
range is 180 to 350 km s$^{-1}$. Our spectroscopic maps span continuously
from the galaxy mid-lane to heights of 2.6 kpc, and sample projected
radii from 0.25 to 11 kpc from the center on {\it both} sides of the
disk. With these data we see clear trends in age with both projected
radius and height.

As a preliminary step in quantifying these age trends, and possible
trends in metallicity and abundance, we undertook measurement of
well-known spectroscopic indices including D$_n$4000 and the Lick
indices associated with Mgb, $<$Fe$>$, and [MgFe]. We have compared
these to standard, solar-abundance SPS models brought forward to the
resolution of the data and constructed for a range of on-going
star-formation. [For the purpose of exploring possible abundance
  variations we have also used models from Worthey for both solar and
  +0.3 dex solar abundances]. Our primary findings are:

% in O, Mg, Si, S, and Ca.

\begin{enumerate}
  \item There is a clear transition with height above NGC 891's disk
    midplane between young and old populations at \val{0.4}{kpc}
    (roughly the broad-band exponential scale-height), consistent with
    models of heating of the stellar in the Milky Way's Solar
    cylinder.

  \item For $|z| > \val{0.4}{kpc}$ there is a trend towards younger
    populations at larger projected radii, consistent with an
    inside-out formation history in NGC 891. The trend also suggests a
    flaring of the young stellar disk at radii beyond 8 kpc, which
    happens to be where \citet{Schechtman-Rook13} found an outer break
    in the super-thin (presumably star-forming) disk.

  \item Beyond 8 kpc in radius and between 0.4 kpc $\leq |z| <$ 1 kpc
    there is a a clear asymmetry in age between the two sides of the
    galaxy. The approaching side, where there is more H$\alpha$
    emission, appears younger. The extent to which this is an m=1
    asymmetry rather than a differential line-of-sight depth effect
    due to the (m=2) geometric arrangement of dust trailing stars and
    gas in spirals arms (as argued by \citet{Kamphuis07b} on the basis
    of 24$\mu$m emission) is not yet clear.  The gist of the argument
    against lopsided variations would be that the young-disk flaring
    between heights of 0.4 to 1 kpc is not as clearly seen on the
    receding side because these young stars are in a spiral arm
    projected behind a relative thick dust layer.  This suggests that
    better quantification of line-of-sight depth is needed to
    accurately interpret this asymmetry in the variation in vertical
    structure {\it within} NGC 891's disk.

\end{enumerate}

Secondary findings include there is little evidence for mean
metallicities outside the range of 0.2-2.5 \Zsol or abundances
outside the range of 0 to +0.3 dex of solar values. Changes in
metallicity and abundance are more difficult to quantify with these
small sets of indices, particularly given the relatively young ages at
low heights. There is a hint that metallicity increases slightly at
intermediate height, but we caution this may be modulated by
projection effects due to a negative metallicity gradient with radius
and increasing line-of-sight depth with height. There is evidence for
metallicity differences at the {\it same} height (above 0.4 kpc) and
{\it projected} radius (between 3 and 8 kpc) on either half of the
galaxy. Again, this may be a projection effect due to different
line-of-sight depths along the two halves of the galaxy, and in that
sense it may be related to the asymmetry seen in age versus height
beyond 8 kpc. However, if there is indeed a negative radial
metallicity gradient this would require the line-of-sight depth for
older stars to be somewhat larger on the receding side, which is
plausible given they will not be preferentially in spirals arms.

%   \item An increase in abundance and decrease in total metallicity
%     with height, consistent with measurements of the Milky Way and
%     nearby galaxies.

We stress that all of these measurements are {\it light}-weighted by
the very virtue of their observational nature, and hence are sensitive
to star-formation history (as we would like), and line-of-sight
depth. The latter is a considerable complication for edge-on galaxies,
and must be overcome for a detailed picture of the location of
different stellar populations in NGC 891 to emerge. The index
measurements reported above provide powerful qualitative measurements
of the general trends in stellar populations in NGC 891, but a
detailed quantitative assessment is hampered by degeneracies in age
and metallicity (see, for example, Figure \ref{891_1:fig:D4000_cuts}).
In Paper II we employ full spectra fitting, guided by the results from
these indices, to reduce this degeneracy and make better quantitative
measurements of the population age as a function of both radius and
height. These measurements include extinction estimates and,
critically, kinematic estimates line-of-sight depth to our data in an
effort to de-project our radial measurements. What is robust from
these preliminary measurements, however, is the presence of vertical
age gradients in NGC 891 that appear much like what we see in the
Milky Way's disk.



\chapter{$\tau$ Model of Star Formation}
\label{chap:tau_model}
Our galaxy models are constructed from the simple stellar population
(SSP) library of \citet{Bruzual03} with an initial mass function from
\citet{Chabrier03}. These SSPs rely on the STELIB stellar library
\citep{LeBorgne03}. The goal of any regularized galaxy model is to
assign individual SSP weights based on a modeled star formation
history (SFH). Our SSP basis set is normalized to \val{1}{M_{\odot}}
and thus the SSP weights will be the total mass of stars in a give
population. We let the mass of a model galaxy at lookback time, $t$,
be
\begin{equation}
\label{TM:eq:M(t)}
M(t) = \int_{t_{\rm form}}^t \psi(t') dt',
\end{equation}
where $\psi(t)$ is the amount of mass in stars produced at loockback time
$t$. Equation \ref{TM:eq:M(t)} is generally true for any star formation
history. For a $\tau$ model we use
\begin{equation}
\label{TM:eq:taumodel}
\psi(t) = \psi_0 e^{-t/\tau_{SF}}.
\end{equation}

Taking the above, our galaxy flux, $G(\lambda)$, is constructed by
weighting the spectra of individual SSPs, $f_i(\lambda)$, by the mass
formed during their formation age:
\begin{equation}
G(\lambda) = \sum_i^N f_i(\lambda) M_i,
\end{equation}
where
\begin{equation}
M_i = \int_{t_2}^{t_1} \psi(t') dt'.
\end{equation}

In practice our SSP ages are discreet and we need to choose a range of time
($t_2 - t_1$) over which SSP contributes to star formation. We set these
limits of integration such that an equal amount of mass is formed in each half
of a logarithmically-spaced SSP age bin. In other words, given an SSP of age
$t_i$, the limits of integration are
\begin{equation}
t_{2,i} = \frac{\log (t_{i+1}) - \log (t_i)}{2} = t_{1,i+1}.
\end{equation}
We set the end points $t_{1,0} = 0$ and $t_{2,N} = t_{\rm form}$.

In the situation where $t_{\rm form}$ is less than the oldest SSP in
our library the SSP mass bins will be unevenly spaced in age. To fix
this problem we assign each SSP an age that is weighted by the amount
of mass formed during the time assigned to its bin. In other words,
the weighted age of an SSP with age $t_i$ is the midpoint of mass bin
$M_i$. This weighted age, $t_{i,w}$, is defined such that
\begin{equation}
\int_{t_1}^{t_{i,w}} \psi(t') dt' = \int_{t_{i,w}}^{t_2} \psi(t') dt'.
\end{equation}
If we assume the $\tau$ model described in Eq. \ref{TM:eq:taumodel} the weighted age is
\begin{equation}
t_{i,w} = \tau_{SF} \log\left( 0.5 \left( e^{t_1/\tau_{SF}} + e^{t_2/\tau_{SF}} \right)\right).
\end{equation}

Finally, we adopt the extinction law of \citet{Charlot00} with a value
of $A_V=1.63$, which defines a reddening term that is constant across all SSPs, 
\begin{equation}
R(\lambda) = e^{-\frac{A_V}{1.086} \left(\lambda/\val{5500}{\AA}\right)^{-0.7}},
\end{equation}
 so our final galaxy is given by
\begin{equation}
G(\lambda) = R(\lambda)\sum_i^N f_i(\lambda) M_i.
\end{equation}

As a proxy for star formation history we define the mean light-weighted age as
\begin{equation}
\label{TM:eq:MLWA}
\tau_L = \frac{\sum_k\left[S(\lambda_K) R(\lambda_k) \sum_i f_i(\lambda_k) M_i t_i\right]}{\sum_{k}\left[S(\lambda_k) R(\lambda_k) \sum_i f_i(\lambda_k) M_i\right]}
%% \tau_L = \frac{\sum_k S(\lambda_k)\left(\frac{\sum_{i,j} \psi(t_i,Z_j) f(\lambda, t_i,Z_j) t_i}{\sum_{i,j} \psi(t_i,Z_j) f(\lambda, t_i, Z_j)} \right)}{\sum_k S(\lambda_k)},
\end{equation}
where $S(\lambda_j)$ defines the bandpass over which the age is computed. We
set $S(\lambda_j)$ to be flat over \val{5450}{\AA} $\leq\lambda\leq$
\val{5550}{\AA} and zero everywhere else.

\bibliographystyle{thesis}
\bibliography{ms_n891_paper}

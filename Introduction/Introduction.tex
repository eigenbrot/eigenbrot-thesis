\chapter[Introduction]{Introduction}
\label{chap:intro}

% Leave space between title and quote or publication note.  This has often been
% 10cm for a quote and 8 cm for a reference, but this is really up to you.
%\vspace{8cm}

%\vfil\eject\clearpage
\clearpage
It has long been known that stars in the solar neighborhood have
vertical scale-heights and velocity dispersions that increase with age
\citep[e.g.,][]{Wielen74}, but the origin of the heating process has
never been settled. Unfortunately, our empirical constraints on disk
heating come only from the solar cylinder in the Milky Way and a few
crude measurements of low-mass edge-on spirals close enough for HST
star-counts \citep{Seth05a}. This scant data is insufficient to resolve
several outstanding questions: How does the heating rate evolve with
time and radius?  Why does disk heating in the Milky Way appear to
saturate after about 5 Gyr?  What, then, gives rise to the thick disk?
And is heating in the Milky Way representative of disk galaxies in
general? Answers to these questions have critical implications for our
picture of disk evolution, all the more poignant given recent
observational claims that high-redshift ionized gas disks are
dynamically hot, clumpy, and thick \citep{Forster-Schreiber09} and
models \citep{Bird13} that predict the formation of observed disk
structure from these high-redshift disks.

Clearly more data is needed, and the rise of resolved spectroscopic
surveys of external galaxies (e.g., MaNGA, SAMI, CALIFA) will offer a
new window into the distribution of stellar populations outside the
Milky Way. However, these surveys lack the spatial resolution
necessary to make detailed comparisons to the substructures seen in
the Milky Way and it is crucial to complement their broad scope with
focused measurements of nearby galaxies. NGC 891 offers an attractive
target for such a study; it is nearby, almost perfectly edge-on, and
is thought to be a good Milky Way analog. In this thesis I present a
study of stellar populations in NGC 891, specifically where these
populations fall in a six dimensional space containing three position
dimensions along with age, metallicity, and extinction.

The rest of this chapter is organized as follows: in
\S\ref{intro:sec:MW} I review the picture of disk formation as
revealed locally in the Milky Way, in \S\ref{intro:sec:SSP} I discuss
the details of the methods (namely full-spectral fitting) that will be
used to measure population in NGC 891, and in \S\ref{intro:sec:fiber}
I discuss the background of fiber integral field units (IFU) and lay
the groundwork for the introduction of \GP; the world's first
dual-head, variable-pitch IFU.

\section{Stellar Populations in the Milky Way}

By the middle of the last century it was well established that the
scale-heights and velocity dispersions of stars in the solar
neighborhood increase with age \citep[see][for a summary of this early
  work, particularly the chapters contributed by Elvius and
  Delhaye]{Blaauw65}. The seminal work by \citet{Roman50} demonstrated
that the disk kinematics also depended on metallicity.  Today these
patterns are known in the literature on Galactic archaeology as
age-velocity-metallicity (abundance) relations \citep[AVM$\alpha$-R;
  e.g.,][]{Aumer09,Minchev14}. Observational advances continued for
the solar neighborhood \citep[e.g.,][]{Edvardsson93, Dehnen98,
  Nordstrom04}, and by the beginning of this century the complexity of
these relations have been mapped throughout much of Milky Way (MW) by
wide-field spectroscopic surveys (e.g., RAVE, \citealt{steinmetz06a};
BRAVA, \citealt{howard08a}, SEGUE, \citealt{yanny09a}, LAMOST,
\citealt{zhao12a} GALAH, \citealt{desilva15a},
Gaia-ESO,\citealt{gilmore12a}; and APOGEE-1 and -2,
\citealt{Majewski15}). The radial gradients in these relations are
beautifully shown by, e.g., by \citet{Bovy12c} and \citet{Hayden15},
illustrating both metallicity and abundance as well-known
complementary chemical-evolutionary tracers. Despite a century of
remarkable progress, two broad but intertwined questions remain: (i)
What are the astrophysical processes (i.e., the chemo-dynamical
explanation) leading to the observed relations; and (ii) are these
patterns generic for spiral disks or specific to the Milky Way?

Setting aside chemical evolution for simplicity, there has been a
long-standing debate about the origin of the vertical stratification
of disk stars in phase-space as a function of age (the age-velocity
relation, or AV-R). Historically the argument has been in the context
of dynamical heating from two-body scattering \citep{Spitzer51}, but
the scattering source has been debated \citep[e.g., giant molecular
  clouds, transient spiral structure, or dwarf satellite
  galaxies][]{Spitzer51,Spitzer53,Wielen77,Quinn93,Binney00}, and none
have proven satisfactory to explain the MW's thick disk.  This
framework has been salvaged but also up-ended by relatively recent
evidence for the increasing turbulence (and presumably thickness) of
ionized gas in disks at higher redshift
\citep{Weiner06,Forster-Schreiber09,Wisnioski15}. It seems plausible
that early phases of disk formation involved gas cooling, leaving
behind an old thick-disk stellar component
\citep{Brook04,Bournaud09}. However, thinner relic layers would also
emerge as time progressed \citep{Bird13}, depending critically on the
cooling time-scale for the gas in the presence of star-formation, AGN
feedback, and accretion, but without invoking any need for dynamical
heating. Ironically, this `settling' of the stellar disk with
population age is not unlike the predictions of monolithic collapse
from \citet{ELS}, albeit now consistent in the context of bottom-up,
or hierarchical structure formation as seen in recent simulations
\citep[e.g.,][]{Bird13,Martig14a}.  Because it is no longer clear if,
loosely speaking, disks `heat' or `cool' to form the the vertical
stratification of disk stars in phase-space, and likely both modes
play a role at late and early times, respectively, henceforth we refer
instead to `dynamical stratification' as a phenomenon that captures
both general physical processes.

The recent simulations noted above show there is a rich history of
radial and vertical build up of stellar populations that involves and
interplay between the cooling of the gas, the impact of mergers and
accretion, and at late times the classical heating processes noted
above.  This richness suggests the possibility for a diversity of
astrophysical paths in disk formation that could lead to significantly
different structure in galaxies, exhibited in their
AVM$\alpha$-Rs. Hence, the broader question of whether the MW is
representative of the external disk galaxy population becomes salient.

%% Maybe move to previous section
Little is known about the dynamical stratification rates for stars in
spiral outside the Milky Way, but recent studies of stellar
populations and dynamical stratification in low-mass spiral galaxies
\citep{Seth05a,Bernard15} have shown dramatic differences in the
age-metallicty and age-velocity dispersion relations when compared to
the Milky Way. Recent measurements of the stellar velocity dispersions
in M31 \citep{Dorman15} show that there are also gradients in
dispersion with age and metallicity, but the amplitudes and
time-scales are larger than in the MW. Differences in velocity
dispersion amplitudes may reflect a more massive or thinner M31 disk,
but possibly also a different dynamical history -- for gas settling or
stellar dynamical heating. Clearly more data on the stellar properties
of external galaxies is needed. The above studies serve as a gold
standard since they are based on studies of resolved stellar
populations. Because there are no massive spiral galaxies outside of
the Local Group for which we can resolve stellar populations at
surface-densities high enough to probe most of the disk, it is
imperative to undertake studies based on integrated starlight.

\section{Deriving Population Properties with Full Spectrum Fitting}
In this work we employ full-spectral fitting to measure age,
metallicity, and extinction (our three ``quantities of interest'') as
a function of radius and height in NGC 891. The idea of using an
entire galaxy spectrum has a long history since \citet{Spinrad71}
first mixed stars together and modern techniques employ a variety of
sophisticated methods to extract the maximum about of data from each
wavelength. Regardless of specific method, all attempts at
full-spectrum fitting require the same basic ingredients: (i) a
library of stellar spectra, whether empirical or synthetic, (ii) a set
of isochrones that encapsulate how stars evolve with time, and (iii)
an initial mass function (IMF). With these three components
Astronomers can construct simple stellar populations (SSPs); a set of
stars of the same age and same metallicity/abundance with a mass
distribution determined by the IMF. To simulate an entire galaxy
multiple SSPs of different ages and metallicities are combine together
to produce a complext stellar population (CSP), which requires
assuming both a star formation history (SFH) and the
distribution/properties of dust in the galaxy. An excellent discussion
of this process can be found in \citet[especially his Figure
  1]{Conroy13}.

Within the general picture painted above there exists a wide range of
options and data. It is common for SSP libraries to be constructed
with the Padova isochrones \citep{Bertelli94, Girardi00, Marigo08}
because these models cover the widest range of stellar age and
chemical compositions, but other models are often used for their focus
on specific epochs of stellar evolution, for example high-mass stars
(Geneva \citep{Schaller92,Meynet00}), low-mass stars ($Y^2$
\citep{Yi01,Yi03}, or Dartmouth \citep{Dotter08}), and even very
low-mass stars (Lyon \citep{Chabrier97,Baraffe98}). In this work we
use exclusively the Padova isochrones because our observations, by
their very nature, light-weighted and therefore the specific details
of low mass stars are rather unimportant.

The choice of IMF can also affect the final modeled galaxy
spectrum. The canonical IMF of \citet{Salpeter55} was based on
observations of the Solar Cylinder in the Milky Way, and there is so
far little evidence that the IMF is appreciably different elsewhere in
the Universe \citep{Bastian10}. More recent observations have refined
the specific form of the IMF \citep{Kroupa01, Chabrier03}, but the
general picture remains the same. We use the IMF of \citet{Chabrier03}
because it is physically motivated and provides a good fit to low-mass
and brown dwarf star counts in the Milky Way
\citep{Bruzual03,Chabrier01,Chabrier03}.

Finally, the construction of SSPs depends on the stellar library
used. In this work we consider only empirical stellar libraries, and
more specifically on the STELIB \citep{LeBorgne03} and MILES
\citep{Sanchez-Blazquez06} libraries. The main strength of emperical
libraries is that the get the chemistry right by default, as indeed
they must. The cost of this accuracy, however, is a very limit
sampling of the entire parameter space of stellar evolution. For
example, as is discussed in \S\ref{sec:ma11}, the MILES library very
coarsly samples the metallicty/age plane and doesn't have any spectra
for ages below \val{6.5}{Myr}. We use the STELIB library because it
more finely samples stars of different ages, but warn that it still
lacks a detailed view of how spectra change with metallicity.

As discussed in \S\ref{sec:SSP_sets} we ultimately use the SSPs of
\citet{Bruzual03}, which are constructed with the Padova isochrones,
Chabrier IMF, and STELIB library. We note, however, that over the
wavelength range we consider ($\val{3800}{\AA} \leq\lambda\leq
\val{6800}{\AA}$) the differences in spectral shape caused by
different assumptions/models are minimal.

More directly relevant to our work is the assumption about the SFH
that is used to construct galaxies (CSPs) from SSPs. A common choice
for SFH is the so called $\tau$-model where the star formation rate
(SFR) follows an exponential function with a single scale parameter,
$\tau_\mathrm{SF}$. This analytic form is based on closed-box models
where the SFR depends linearly on gas density \citep{Schmidt59} and
offers an attractive, one parameter, parameterization of the SFH. In
this work we chose to use a non-parametric SFH (as discussed in
\S\ref{sec:SSP_method}) which allows us to reduce the systematics in
our results that arise from forcing the form of the SFR (systematics
are not completely eliminated, however, as discussed in
\S\ref{sec:sys_err}). Using this method results in a much larger set
of free parameters and puts more strain on the fitting code, but our
data have high enough signal to noise (\val{\asim 60}{\AA^{-1}}) to
make it a viable option.

Once model galaxies are constructed there are a multidue of methods
available to fit them to our data, for example those of
\citep{Cappellari04,Tojeiro07,Chen12,CidFernandes05,Ocvirk06,Wilkinson15,Sanchez16}. Regardless
of the method used these fits face the same common issues; namely how
to deal with known degeneracies between age, metallicity, and
extinction \citep{Oconnel76,Aaronson78,Worthey94,dePaz02}. In some
methods the extinction degeneracy can be mitigated by removing the
overall continuum from both the data and models before fitting
\citep[e.g.,][]{Ocvirk06,Wilkinson15} and then recovering an
extinction estimate either by measurements of gass emission (i.e., the
Balmer decrement) or separate analysis of the ``best fit'' galaxy
spectrum. Metallicity and age are more closely entwined and the
degeneracy between them more difficult to break. In this work we
attempt to quantify the uncertainties that arise from similarities
between SSPs that are degenerate with age and metallicity (see
\S\ref{sec:fit_err}), but note that we have not addressed systematic
uncertainties that may arise from our choice of model SSPs.

\section{A Fiber Optic Primer}

The use of fused silica optical fibers for astronomical observations
was first suggest by \citet{Angel77}, and in the intervening decades
their importance and usefulness to Astronomy has only increased. The
astronomical benefits of fiber optics are essentially two fold:
Firstly, they allow instruments to be decoupled from the telescope
focal plane, which enables the construction of very large and
sensitive spectrographs that are free from the unstable environmental
conditions often found on the oberving floor of most
telescopes. Secondly, they can be easily placed at an arbitrary
location on the sky while maintaing a consistent spectrograph input.

A class of fiber optic instruments called integral field units (IFUs)
make great use of this second point. IFUs are a collection of fibers
that are placed in some two-dimensional configuration on the sky and
therefore produce data that exist in three dimensions (two spatial and
one spectral). Some IFUs are designed to efficiently measure the
spectra of a large field of stars, for example HYDRA {\bf REF? Can't
  find it!}, and can have the location of their fibers changed from
program to program. Others have a fiber layout that is fixed and
usually intented for observations of extended objects. SparsePak
\citep{Bershady04,Bershady05} is an excellent example of this
type. The trade off for the lack of reconfigurability in fixed IFUs is
a generally tighter fiber packing and therefore improved spatial
resolution.

In the past, difficulties in constuction resulted in a cottage
industry of IFU builders, but more recently there has been an
explosion of mass-produced IFUs that have allowed large, resolved
spectrogrpahic surveys like MaNGA \citep{Bundy15}, SAMI \citep{Croom12},
and CALIFA \citep{Sanchez12} to rapidly expand our view of the Universe. In
this thesis I present \GP and HexPak, a set of IFUs that are the first
in the world to contain fibers of different sizes which allows them to
be tailored to specific scientific goals.

With great power comes great responsibility, however, and fiber optics
affect the light passing through them in ways that have profound
implications on data quality and instrument design. In particular,
fiber optics not only attenuate precious astronomical data, but also
increase the entropy in the beam. The latter effect is referred to as
focal ratio degradation (FRD), whereby light injected into a fiber at
a particular angle emerges from the fiber at a larger angle. This
increase in entropy creates a need for larger (and more expensive)
spectrograph optics and can decrease the total system throughput if
not properly accounted for. An understanding of the causes of FRD can
help mitigate its effects and the first theories placed the blame on
microbends along the length of the fiber
\cite{Gloge72,Carrasco94}. Recent studies have suggested however, that
most FRD is caused by scattering at the surface of the fiber
\citep{Avila98,Haynes11,Eigenbrot12}. In the latter scenario it is
likely that surface treatments (i.e., anti-reflective coatings) can
mitigate the effects of FRD.

\clearpage
\phantomsection % Fixes references link in hyperref/PDF index

% Requires thesis.bst to be present (or linked) in chapter subdirectory.
\bibliographystyle{thesis}
\bibliography{Introduction}

\chapter[Using \GP]{Observing NGC 891 with \GP: Novel Observational and Data Reduction Techniques}
\label{chap:gradpak_obs}

% Leave space between title and quote or publication note.  This has often been
% 10cm for a quote and 8 cm for a reference, but this is really up to you.
%\vspace{8cm}

%%%%%%%%%%%%%%%%%%%%%%%%%%%%%%%%%%%%%%%%%%%%%%%%%%%%%%%%%%%%% 
%% \begin{chabstract}
%%     Chapter abstract.
%% \end{chabstract}
%% \cleardoublepage

\section{Outline}
Using \GP presented its own unique set of challenges. In this chapter I will
detail the observation and reduction techniques I developed to get useful data
from this new IFU.

One question I have is where to put my GradPak\_guide. Much of the information
in that document will be reused in this chapter, but some of the lower-level
items (e.g. program API) seem out of place in a thesis. The whole guide will
live online somewhere and maybe the lower-level stuff can go in an appendix or
something, but I'm not so sure.

This chapter will also detail the observing program that resulted in our set
of NGC 891 data. Things like observing logs, pointings, etc. will go here. As
the end product of both the new reduction techniques and the NGC 891 program
we will show some reduced spectra and make some broad scientific
interpretations.

\section{Schedule}
Much of this writing is already done. More importantly, all of the
developement/analysis is complete. I estimate a week of focused effort is more
than enough to finish this chapter.

\bibliographystyle{thesis}
\bibliography{GradPak_obs}

\chapter*{Abstract}
\addcontentsline{toc}{section}{Abstract}
\vspace{-0.5in}
Studies of the Milky Way show that the vertical velocity dispersion of
stars in the disk tends to increase with age. However it is not clear
how this ``heating'' ties into the formation of the observed Milky Way
disk structure, or disk galaxies in general. In this thesis I present
a study of stellar populations in NGC 891 that adds information to our
information on the structure of galaxies outside the Milky Way.

This study was enabled by the construction of HexPak/\GP, two fiber
integral field units that together form the worlds first dual-head,
multi-pitch fiber instrument. The unique design of \GP allowed for the
rapid collection of high quality data over a large range in surface
brightness and at a high filling factor.
% When designing HexPak/\GP I investigated sources of fiber focal
% ratio degredation and found that surface scattering likely plays a
% large role in the degredation of light emitted by fiber optics.

Using \GP I measure trends in stellar population as revealed by both
spectral indices and full-spectrum fitting. I find that the overall
picture of disk heating in NGC 891 is remarkably similar to trends
measured in the Milky Way; the presence of young populations is
abruptly cut off at \asim 1 scale height. Furthermore I find that
populations generally get younger farther from the center of NGC 891,
perhaps the signature of an inside-out formation scenario. It also
appears that each generation of stars forms a flared disk, with later
generations forming flares at larger radii.

Finally, I identify two distinct epochs in the enrichment history of
NGC 891: (i) a period of self-enrichment lasting from galaxy formation
until \val{\asim 6}{Gyr} ago, and (ii) a period of decreasing
metallicity caused by the infall of pristine gas from outside the
galaxy that began \val{6}{Gyr} ago and continues today.
